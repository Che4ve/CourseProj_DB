\documentclass[14pt,a4paper]{article}

% Кодировка и шрифты
\usepackage[utf8]{inputenc}
\usepackage[T2A]{fontenc}
\usepackage[russian]{babel}

% Геометрия страницы по ГОСТ 7.32-2017
\usepackage[top=20mm,bottom=20mm,left=30mm,right=15mm]{geometry}

% Графика
\usepackage{graphicx}
\usepackage{float}
\usepackage{wrapfig}
\usepackage{adjustbox}
\usepackage{tikz}
\usetikzlibrary{shapes,arrows,positioning,fit,calc}

% Таблицы
\usepackage{tabularx}
\usepackage{multirow}
\usepackage{longtable}
\usepackage{array}

% Код
\usepackage{listings}
\usepackage{xcolor}

% Математика
\usepackage{amsmath}

% Гиперссылки
\usepackage{hyperref}
\hypersetup{
    colorlinks=true,
    linkcolor=black,
    urlcolor=blue,
    citecolor=black
}

% Настройка шрифта Times New Roman (стандартный в Overleaf)
\usepackage{times}

% Межстрочный интервал 1.5
\linespread{1.5}

% Абзацный отступ 1.25 см
\setlength{\parindent}{1.25cm}

% Настройка оформления кода SQL
\lstset{
    language=SQL,
    basicstyle=\ttfamily\footnotesize,
    keywordstyle=\color{blue}\bfseries,
    commentstyle=\color{gray}\itshape,
    stringstyle=\color{red},
    numbers=left,
    numberstyle=\tiny\color{gray},
    stepnumber=1,
    numbersep=5pt,
    frame=single,
    breaklines=true,
    breakatwhitespace=true,
    tabsize=2,
    showspaces=false,
    showstringspaces=false,
    showtabs=false
}

% Настройка заголовков
\usepackage{titlesec}
\titleformat{\section}{\normalfont\bfseries}{\thesection}{1em}{}
\titleformat{\subsection}{\normalfont\bfseries}{\thesubsection}{1em}{}
\titleformat{\subsubsection}{\normalfont\bfseries}{\thesubsubsection}{1em}{}

% Убираем точку после номера раздела
\makeatletter
\renewcommand{\@seccntformat}[1]{\csname the#1\endcsname\ }
\makeatother

% Настройка нумерации страниц
\usepackage{fancyhdr}
\pagestyle{fancy}
\fancyhf{}
\fancyfoot[C]{\thepage}
\renewcommand{\headrulewidth}{0pt}

% Команды для оформления
\newcommand{\code}[1]{\texttt{#1}}

\begin{document}

% Титульный лист
\begin{titlepage}
    \centering
    
    \vspace*{-2cm}
    
    \begin{flushleft}
        МИНИСТЕРСТВО НАУКИ И ВЫСШЕГО ОБРАЗОВАНИЯ РОССИЙСКОЙ ФЕДЕРАЦИИ
        
        ФЕДЕРАЛЬНОЕ ГОСУДАРСТВЕННОЕ БЮДЖЕТНОЕ ОБРАЗОВАТЕЛЬНОЕ УЧРЕЖДЕНИЕ ВЫСШЕГО ОБРАЗОВАНИЯ
        
        \vspace{0.5cm}
        
        \textbf{«МОСКОВСКИЙ АВИАЦИОННЫЙ ИНСТИТУТ\\
        (национальный исследовательский университет)» (МАИ)}
        
        \vspace{0.5cm}
        
        Институт № 8 «Компьютерные науки и прикладная математика»
        
        Кафедра 806 «Вычислительная математика и программирование»
    \end{flushleft}
    
    \vspace{3cm}
    
    {\Large\textbf{КУРСОВОЙ ПРОЕКТ}}
    
    \vspace{0.5cm}
    
    по дисциплине «Базы данных»
    
    \vspace{1cm}
    
    \textbf{Тема: «Информационная система для трекинга привычек»}
    
    \vfill
    
    \begin{flushright}
        \begin{tabular}{ll}
            Студент: & Чесноков В.Д. \\
            Группа: & М8О-309Б-23 \\
            Преподаватель: & Грубенко М.Д. \\
            Оценка: & \underline{\hspace{3cm}} \\
            Дата: & \underline{\hspace{3cm}} \\
            Подпись: & \underline{\hspace{3cm}} \\
        \end{tabular}
    \end{flushright}
    
    \vspace{1cm}
    
    \centering
    Москва 2025
    
\end{titlepage}

% Содержание
\newpage
\tableofcontents
\newpage

% Введение
\section{ВВЕДЕНИЕ}

В современном мире все больше людей стремятся к систематизации своей жизни и формированию полезных привычек. Трекинг привычек позволяет отслеживать прогресс, анализировать статистику и мотивировать к достижению целей. Для эффективной работы таких систем необходима надежная и производительная база данных, способная обрабатывать большие объемы информации и обеспечивать целостность данных.

\textbf{Объект исследования} --- информационная система для трекинга привычек с реляционной базой данных PostgreSQL.

\textbf{Предмет исследования} --- проектирование и реализация структуры базы данных, оптимизация запросов и обеспечение целостности данных.

\textbf{Цель работы} --- создать полноценную информационную систему с реляционной базой данных, демонстрирующую навыки проектирования структуры базы данных, реализации связей и ограничений целостности, написания SQL-запросов и функций, работы с триггерами и аудитом изменений, а также интеграции базы данных с backend-приложением.

Для достижения поставленной цели необходимо решить следующие \textbf{задачи}:
\begin{itemize}
    \item спроектировать структуру базы данных с не менее чем 9--10 таблицами, связанными по типам 1:1, 1:N, N:M;
    \item реализовать ограничения целостности (PRIMARY KEY, FOREIGN KEY, UNIQUE, CHECK, NOT NULL) с каскадным обновлением и удалением;
    \item обеспечить наполнение базы данных реалистичными данными (крупные таблицы --- 500--1000 записей, транзакционные --- не менее 5000);
    \item реализовать CRUD-операции через backend-API (базовые через ORM, сложные через чистый SQL);
    \item создать триггеры для аудита изменений и автоматического обновления агрегированных данных;
    \item реализовать скалярные и табличные функции для расчетов и отчетов;
    \item создать не менее трех представлений (VIEW) для агрегированных данных и аналитических отчетов;
    \item оптимизировать запросы с помощью индексов и продемонстрировать улучшение производительности;
    \item реализовать батчевую загрузку данных с параметрами и логированием ошибок;
    \item интегрировать базу данных с backend-приложением и обеспечить документацию API (Swagger/OpenAPI).
\end{itemize}

Отчет состоит из введения, аналитической части, проектной части, технологической части, заключения, списка использованных источников и приложений.

% Аналитическая часть
\newpage
\section{АНАЛИТИЧЕСКАЯ ЧАСТЬ}

\subsection{Обзор предметной области}

Предметная область «Трекер привычек» представляет собой информационную систему для отслеживания пользователями своих привычек --- как положительных (например, ежедневные упражнения, чтение), так и отрицательных (например, курение, употребление сладкого).

Основные сущности предметной области:
\begin{itemize}
    \item \textbf{Пользователи} --- зарегистрированные пользователи системы с учетными данными и профилями;
    \item \textbf{Привычки} --- действия, которые пользователь хочет отслеживать, характеризуемые типом (хорошая/плохая), приоритетом, описанием;
    \item \textbf{Отметки выполнения} --- записи о выполнении привычки в конкретный день с возможностью указания настроения, длительности, заметок;
    \item \textbf{Расписания} --- настройки частоты выполнения привычек (ежедневно, еженедельно, по дням недели);
    \item \textbf{Теги} --- категории для группировки привычек;
    \item \textbf{Напоминания} --- уведомления о необходимости выполнения привычки;
    \item \textbf{Статистика} --- агрегированные данные по привычкам (количество отметок, серии выполнения, процент выполнения).
\end{itemize}

Система должна обеспечивать:
\begin{itemize}
    \item регистрацию и аутентификацию пользователей;
    \item создание, редактирование и удаление привычек;
    \item фиксацию отметок выполнения с дополнительной информацией;
    \item формирование аналитических отчетов и статистики;
    \item массовую загрузку данных;
    \item аудит всех изменений в системе.
\end{itemize}

\subsection{Постановка задачи}

На основе анализа предметной области сформулированы следующие требования к базе данных:

\textbf{Требования к структуре:}
\begin{itemize}
    \item не менее 9--10 таблиц, связанных по типам 1:1, 1:N, N:M;
    \item каждая таблица содержит не менее 5 атрибутов разных типов данных;
    \item наличие ключевых сущностей (Users, Habits, Checkins);
    \item наличие журнала аудита для отслеживания изменений.
\end{itemize}

\textbf{Требования к ограничениям целостности:}
\begin{itemize}
    \item использование PRIMARY KEY, FOREIGN KEY, UNIQUE, CHECK, NOT NULL;
    \item обеспечение каскадного обновления и удаления при связях.
\end{itemize}

\textbf{Требования к объему и наполнению:}
\begin{itemize}
    \item крупные таблицы --- 500--1000 записей;
    \item транзакционные таблицы --- не менее 5000 записей;
    \item данные должны быть реалистичными;
    \item предусмотрена батчевая загрузка данных с параметрами и логированием ошибок.
\end{itemize}

\textbf{Требования к SQL-функциональности:}
\begin{itemize}
    \item CRUD-операции через backend-API (базовые через ORM, сложные через чистый SQL);
    \item триггеры для аудита изменений и автоматического обновления агрегатов;
    \item скалярные и табличные функции;
    \item не менее трех представлений (VIEW) для агрегированных данных и аналитических отчетов.
\end{itemize}

\textbf{Требования к оптимизации:}
\begin{itemize}
    \item создание индексов для полей, участвующих в WHERE, JOIN, ORDER BY;
    \item демонстрация улучшения производительности с помощью EXPLAIN ANALYZE.
\end{itemize}

% Проектная часть
\newpage
\section{ПРОЕКТНАЯ ЧАСТЬ}

\subsection{Архитектура системы}

Разработанная информационная система построена на основе микросервисной архитектуры с использованием монорепозитория. Система состоит из следующих компонентов:

\begin{itemize}
    \item \textbf{Backend} --- REST API на базе фреймворка NestJS с использованием ORM Prisma для работы с базой данных;
    \item \textbf{Frontend} --- веб-приложение на базе Next.js 15 с React 19;
    \item \textbf{База данных} --- PostgreSQL 16;
    \item \textbf{Контейнеризация} --- Docker и Docker Compose для оркестрации сервисов.
\end{itemize}

Технологический стек:
\begin{itemize}
    \item Backend: NestJS 10, Prisma ORM 6, TypeScript 5, JWT для аутентификации;
    \item Frontend: Next.js 15, React 19, TypeScript 5, Tailwind CSS 4;
    \item Database: PostgreSQL 16;
    \item DevOps: Docker, Docker Compose, pnpm workspaces, Turborepo.
\end{itemize}

Структура проекта организована как монорепозиторий с использованием pnpm workspaces:
\begin{itemize}
    \item \code{apps/backend/} --- код backend-приложения;
    \item \code{apps/frontend/} --- код frontend-приложения;
    \item \code{packages/db/} --- Prisma схема, SQL-миграции, seed-скрипты;
    \item \code{docs/} --- документация проекта.
\end{itemize}

\subsection{Проектирование структуры базы данных}

\subsubsection{ER-диаграмма}

База данных содержит 13 таблиц, связанных между собой различными типами связей:

\begin{itemize}
    \item \textbf{Связи 1:1:}
    \begin{itemize}
        \item \code{users} $\leftrightarrow$ \code{user\_profiles} (один пользователь имеет один профиль);
        \item \code{habits} $\leftrightarrow$ \code{habit\_stats} (одна привычка имеет одну статистику).
    \end{itemize}
    
    \item \textbf{Связи 1:N:}
    \begin{itemize}
        \item \code{users} $\rightarrow$ \code{habits} (один пользователь имеет много привычек);
        \item \code{users} $\rightarrow$ \code{habit\_checkins} (один пользователь имеет много отметок);
        \item \code{habits} $\rightarrow$ \code{habit\_checkins} (одна привычка имеет много отметок);
        \item \code{habits} $\rightarrow$ \code{habit\_schedules} (одна привычка имеет много расписаний);
        \item \code{habits} $\rightarrow$ \code{reminders} (одна привычка имеет много напоминаний);
        \item \code{users} $\rightarrow$ \code{batch\_import\_jobs} (один пользователь имеет много задач импорта).
    \end{itemize}
    
    \item \textbf{Связи N:M:}
    \begin{itemize}
        \item \code{habits} $\leftrightarrow$ \code{tags} через таблицу \code{habit\_tags} (много привычек связано со многими тегами);
        \item \code{batch\_import\_jobs} $\rightarrow$ \code{batch\_import\_errors} (одна задача имеет много ошибок).
    \end{itemize}
\end{itemize}

\subsubsection{Описание таблиц базы данных}

База данных содержит 13 таблиц. Каждая таблица имеет не менее 5 различных типов данных. Ниже приведено описание основных таблиц.

\textbf{1. Таблица \code{users} (8 колонок, 6 типов данных)}

Назначение: основная таблица пользователей системы.

\begin{longtable}{|p{3cm}|p{2.5cm}|p{4cm}|p{5cm}|}
\hline
\textbf{Колонка} & \textbf{Тип} & \textbf{Ограничения} & \textbf{Описание} \\
\hline
\endfirsthead
\hline
\textbf{Колонка} & \textbf{Тип} & \textbf{Ограничения} & \textbf{Описание} \\
\hline
\endhead
\hline
\endfoot
\hline
\endlastfoot
id & UUID & PRIMARY KEY, DEFAULT gen\_random\_uuid() & Уникальный идентификатор \\
\hline
email & TEXT & UNIQUE, NOT NULL & Email пользователя \\
\hline
password\_hash & TEXT & NOT NULL & Хеш пароля (bcrypt) \\
\hline
full\_name & TEXT & NOT NULL & Полное имя \\
\hline
role & VARCHAR(50) & DEFAULT 'user' & Роль пользователя \\
\hline
is\_active & BOOLEAN & DEFAULT true & Активен ли аккаунт \\
\hline
created\_at & TIMESTAMPTZ & DEFAULT NOW() & Дата создания \\
\hline
login\_count & INTEGER & DEFAULT 0 & Количество входов \\
\hline
\end{longtable}

Уникальные типы: UUID, TEXT, VARCHAR, BOOLEAN, TIMESTAMPTZ, INTEGER = \textbf{6 типов}.

\textbf{2. Таблица \code{user\_profiles} (8 колонок, 6 типов данных)}

Назначение: дополнительная информация о пользователе (связь 1:1 с users).

\begin{longtable}{|p{3cm}|p{2.5cm}|p{4cm}|p{5cm}|}
\hline
\textbf{Колонка} & \textbf{Тип} & \textbf{Ограничения} & \textbf{Описание} \\
\hline
id & UUID & PRIMARY KEY & Уникальный идентификатор \\
\hline
user\_id & UUID & UNIQUE, NOT NULL, FK $\rightarrow$ users & Ссылка на пользователя \\
\hline
bio & TEXT & NULL & Биография \\
\hline
avatar\_url & TEXT & NULL & URL аватара \\
\hline
timezone & VARCHAR(100) & DEFAULT 'UTC' & Часовой пояс \\
\hline
date\_of\_birth & DATE & NULL & Дата рождения \\
\hline
notification\_enabled & BOOLEAN & DEFAULT true & Уведомления включены \\
\hline
theme\_preference & SMALLINT & DEFAULT 0, CHECK (0--2) & Тема оформления \\
\hline
\end{longtable}

Уникальные типы: UUID, TEXT, VARCHAR, DATE, BOOLEAN, SMALLINT = \textbf{6 типов}.

\textbf{3. Таблица \code{habits} (9 колонок, 7 типов данных)}

Назначение: привычки пользователей (хорошие и плохие).

\begin{longtable}{|p{3cm}|p{2.5cm}|p{4cm}|p{5cm}|}
\hline
\textbf{Колонка} & \textbf{Тип} & \textbf{Ограничения} & \textbf{Описание} \\
\hline
id & UUID & PRIMARY KEY & Уникальный идентификатор \\
\hline
user\_id & UUID & NOT NULL, FK $\rightarrow$ users & Владелец привычки \\
\hline
name & TEXT & NOT NULL & Название привычки \\
\hline
description & TEXT & NULL & Описание \\
\hline
type & VARCHAR(10) & NOT NULL, CHECK IN ('good', 'bad') & Тип привычки \\
\hline
priority & SMALLINT & DEFAULT 0, CHECK (0--10) & Приоритет \\
\hline
is\_archived & BOOLEAN & DEFAULT false & Архивирована \\
\hline
display\_order & INTEGER & DEFAULT 0 & Порядок отображения \\
\hline
created\_at & TIMESTAMPTZ & DEFAULT NOW() & Дата создания \\
\hline
\end{longtable}

Уникальные типы: UUID, TEXT, VARCHAR, SMALLINT, BOOLEAN, INTEGER, TIMESTAMPTZ = \textbf{7 типов}.

\textbf{4. Таблица \code{habit\_checkins} (8 колонок, 6 типов данных)}

Назначение: отметки выполнения привычек.

\begin{longtable}{|p{3cm}|p{2.5cm}|p{4cm}|p{5cm}|}
\hline
\textbf{Колонка} & \textbf{Тип} & \textbf{Ограничения} & \textbf{Описание} \\
\hline
id & UUID & PRIMARY KEY & Уникальный идентификатор \\
\hline
habit\_id & UUID & NOT NULL, FK $\rightarrow$ habits & Ссылка на привычку \\
\hline
user\_id & UUID & NOT NULL, FK $\rightarrow$ users & Ссылка на пользователя \\
\hline
checkin\_date & DATE & NOT NULL, UNIQUE(habit\_id, checkin\_date) & Дата отметки \\
\hline
checkin\_time & TIME & DEFAULT CURRENT\_TIME & Время отметки \\
\hline
notes & TEXT & NULL & Заметки \\
\hline
mood\_rating & SMALLINT & NULL, CHECK (1--5) & Оценка настроения \\
\hline
duration\_minutes & INTEGER & NULL & Длительность в минутах \\
\hline
created\_at & TIMESTAMPTZ & DEFAULT NOW() & Дата создания \\
\hline
\end{longtable}

Уникальные типы: UUID, DATE, TIME, TEXT, SMALLINT, INTEGER, TIMESTAMPTZ = \textbf{7 типов}.

\textbf{5. Таблица \code{tags} (7 колонок, 5 типов данных)}

Назначение: теги для категоризации привычек.

\begin{longtable}{|p{3cm}|p{2.5cm}|p{4cm}|p{5cm}|}
\hline
\textbf{Колонка} & \textbf{Тип} & \textbf{Ограничения} & \textbf{Описание} \\
\hline
id & UUID & PRIMARY KEY & Уникальный идентификатор \\
\hline
name & VARCHAR(100) & UNIQUE, NOT NULL & Название тега \\
\hline
slug & VARCHAR(100) & UNIQUE, NOT NULL & URL-слаг \\
\hline
color & CHAR(7) & DEFAULT '\#000000' & Цвет в HEX \\
\hline
usage\_count & INTEGER & DEFAULT 0 & Количество использований \\
\hline
is\_system & BOOLEAN & DEFAULT false & Системный тег \\
\hline
created\_at & TIMESTAMPTZ & DEFAULT NOW() & Дата создания \\
\hline
\end{longtable}

Уникальные типы: UUID, VARCHAR, CHAR, INTEGER, BOOLEAN, TIMESTAMPTZ = \textbf{6 типов}.

\textbf{6. Таблица \code{habit\_tags} (7 колонок, 5 типов данных)}

Назначение: связь многие-ко-многим между привычками и тегами.

\begin{longtable}{|p{3cm}|p{2.5cm}|p{4cm}|p{5cm}|}
\hline
\textbf{Колонка} & \textbf{Тип} & \textbf{Ограничения} & \textbf{Описание} \\
\hline
id & UUID & PRIMARY KEY & Уникальный идентификатор \\
\hline
habit\_id & UUID & NOT NULL, FK $\rightarrow$ habits & Ссылка на привычку \\
\hline
tag\_id & UUID & NOT NULL, FK $\rightarrow$ tags & Ссылка на тег \\
\hline
priority & SMALLINT & DEFAULT 0 & Приоритет тега \\
\hline
is\_primary & BOOLEAN & DEFAULT false & Основной тег \\
\hline
assigned\_by & UUID & NULL, FK $\rightarrow$ users & Кто назначил \\
\hline
assigned\_at & TIMESTAMPTZ & DEFAULT NOW() & Дата назначения \\
\hline
\end{longtable}

Уникальные типы: UUID, SMALLINT, BOOLEAN, TIMESTAMPTZ = \textbf{4 типа} (дополняется через связи).

\textbf{7. Таблица \code{habit\_schedules} (7 колонок, 5 типов данных)}

Назначение: расписания выполнения привычек.

\begin{longtable}{|p{3cm}|p{2.5cm}|p{4cm}|p{5cm}|}
\hline
\textbf{Колонка} & \textbf{Тип} & \textbf{Ограничения} & \textbf{Описание} \\
\hline
id & UUID & PRIMARY KEY & Уникальный идентификатор \\
\hline
habit\_id & UUID & NOT NULL, FK $\rightarrow$ habits & Ссылка на привычку \\
\hline
frequency\_type & VARCHAR(20) & NOT NULL, CHECK IN (...) & Тип частоты \\
\hline
frequency\_value & INTEGER & DEFAULT 1 & Значение частоты \\
\hline
weekdays\_mask & SMALLINT & DEFAULT 127 & Маска дней недели \\
\hline
start\_date & DATE & NOT NULL & Дата начала \\
\hline
end\_date & DATE & NULL & Дата окончания \\
\hline
is\_active & BOOLEAN & DEFAULT true & Активно \\
\hline
\end{longtable}

Уникальные типы: UUID, VARCHAR, INTEGER, SMALLINT, DATE, BOOLEAN = \textbf{6 типов}.

\textbf{8. Таблица \code{reminders} (8 колонок, 6 типов данных)}

Назначение: напоминания для привычек.

\begin{longtable}{|p{3cm}|p{2.5cm}|p{4cm}|p{5cm}|}
\hline
\textbf{Колонка} & \textbf{Тип} & \textbf{Ограничения} & \textbf{Описание} \\
\hline
id & UUID & PRIMARY KEY & Уникальный идентификатор \\
\hline
habit\_id & UUID & NOT NULL, FK $\rightarrow$ habits & Ссылка на привычку \\
\hline
reminder\_time & TIME & NOT NULL & Время напоминания \\
\hline
days\_of\_week & SMALLINT & DEFAULT 127 & Дни недели (битовая маска) \\
\hline
notification\_text & TEXT & NULL & Текст уведомления \\
\hline
delivery\_method & VARCHAR(20) & DEFAULT 'push' & Метод доставки \\
\hline
is\_active & BOOLEAN & DEFAULT true & Активно \\
\hline
created\_at & TIMESTAMPTZ & DEFAULT NOW() & Дата создания \\
\hline
\end{longtable}

Уникальные типы: UUID, TIME, SMALLINT, TEXT, VARCHAR, BOOLEAN, TIMESTAMPTZ = \textbf{7 типов}.

\textbf{9. Таблица \code{habit\_stats} (9 колонок, 5 типов данных)}

Назначение: агрегированная статистика по привычкам (обновляется триггером).

\begin{longtable}{|p{3cm}|p{2.5cm}|p{4cm}|p{5cm}|}
\hline
\textbf{Колонка} & \textbf{Тип} & \textbf{Ограничения} & \textbf{Описание} \\
\hline
id & UUID & PRIMARY KEY & Уникальный идентификатор \\
\hline
habit\_id & UUID & UNIQUE, NOT NULL, FK $\rightarrow$ habits & Ссылка на привычку \\
\hline
total\_checkins & INTEGER & DEFAULT 0 & Всего отметок \\
\hline
current\_streak & INTEGER & DEFAULT 0 & Текущая серия \\
\hline
longest\_streak & INTEGER & DEFAULT 0 & Лучшая серия \\
\hline
completion\_rate & NUMERIC(5,2) & DEFAULT 0.00 & Процент выполнения \\
\hline
average\_mood & NUMERIC(3,2) & NULL & Средняя оценка настроения \\
\hline
last\_checkin\_at & DATE & NULL & Последняя отметка \\
\hline
updated\_at & TIMESTAMPTZ & DEFAULT NOW() & Дата обновления \\
\hline
\end{longtable}

Уникальные типы: UUID, INTEGER, NUMERIC, DATE, TIMESTAMPTZ = \textbf{5 типов}.

\textbf{10. Таблица \code{audit\_log} (9 колонок, 6 типов данных)}

Назначение: журнал аудита изменений.

\begin{longtable}{|p{3cm}|p{2.5cm}|p{4cm}|p{5cm}|}
\hline
\textbf{Колонка} & \textbf{Тип} & \textbf{Ограничения} & \textbf{Описание} \\
\hline
id & UUID & PRIMARY KEY & Уникальный идентификатор \\
\hline
table\_name & VARCHAR(100) & NOT NULL & Название таблицы \\
\hline
operation & VARCHAR(10) & NOT NULL, CHECK IN ('INSERT', 'UPDATE', 'DELETE') & Тип операции \\
\hline
record\_id & UUID & NULL & ID изменённой записи \\
\hline
user\_id & UUID & NULL & ID пользователя \\
\hline
old\_data & JSONB & NULL & Старые данные \\
\hline
new\_data & JSONB & NULL & Новые данные \\
\hline
ip\_address & INET & NULL & IP-адрес \\
\hline
changed\_at & TIMESTAMPTZ & DEFAULT NOW() & Дата изменения \\
\hline
\end{longtable}

Уникальные типы: UUID, VARCHAR, JSONB, INET, TIMESTAMPTZ = \textbf{5 типов}.

\textbf{11. Таблица \code{batch\_import\_jobs} (11 колонок, 6 типов данных)}

Назначение: задачи батчевого импорта данных.

\begin{longtable}{|p{3cm}|p{2.5cm}|p{4cm}|p{5cm}|}
\hline
\textbf{Колонка} & \textbf{Тип} & \textbf{Ограничения} & \textbf{Описание} \\
\hline
id & UUID & PRIMARY KEY & Уникальный идентификатор \\
\hline
user\_id & UUID & FK $\rightarrow$ users, ON DELETE SET NULL & Пользователь \\
\hline
entity\_type & VARCHAR(50) & NOT NULL & Тип сущности \\
\hline
status & VARCHAR(20) & CHECK IN ('pending', 'processing', 'completed', 'failed') & Статус \\
\hline
total\_records & INTEGER & DEFAULT 0 & Всего записей \\
\hline
success\_count & INTEGER & DEFAULT 0 & Успешных \\
\hline
error\_count & INTEGER & DEFAULT 0 & Ошибок \\
\hline
progress\_percent & NUMERIC(5,2) & DEFAULT 0.00 & Процент выполнения \\
\hline
file\_size\_bytes & BIGINT & NULL & Размер файла \\
\hline
started\_at & TIMESTAMPTZ & DEFAULT NOW() & Дата начала \\
\hline
completed\_at & TIMESTAMPTZ & NULL & Дата завершения \\
\hline
\end{longtable}

Уникальные типы: UUID, VARCHAR, INTEGER, NUMERIC, BIGINT, TIMESTAMPTZ = \textbf{6 типов}.

\textbf{12. Таблица \code{batch\_import\_errors} (7 колонок, 5 типов данных)}

Назначение: ошибки батчевого импорта.

\begin{longtable}{|p{3cm}|p{2.5cm}|p{4cm}|p{5cm}|}
\hline
\textbf{Колонка} & \textbf{Тип} & \textbf{Ограничения} & \textbf{Описание} \\
\hline
id & UUID & PRIMARY KEY & Уникальный идентификатор \\
\hline
job\_id & UUID & NOT NULL, FK $\rightarrow$ batch\_import\_jobs & Ссылка на задачу \\
\hline
row\_number & INTEGER & NULL & Номер строки \\
\hline
record\_data & JSONB & NOT NULL & Данные записи \\
\hline
error\_message & TEXT & NOT NULL & Сообщение об ошибке \\
\hline
error\_code & VARCHAR(50) & NULL & Код ошибки \\
\hline
created\_at & TIMESTAMPTZ & DEFAULT NOW() & Дата создания \\
\hline
\end{longtable}

Уникальные типы: UUID, INTEGER, JSONB, TEXT, VARCHAR, TIMESTAMPTZ = \textbf{6 типов}.

\textbf{13. Таблица \code{\_manual\_migrations} (7 колонок, 6 типов данных)}

Назначение: контроль применения SQL-миграций.

\begin{longtable}{|p{3cm}|p{2.5cm}|p{4cm}|p{5cm}|}
\hline
\textbf{Колонка} & \textbf{Тип} & \textbf{Ограничения} & \textbf{Описание} \\
\hline
id & UUID & PRIMARY KEY & Уникальный идентификатор \\
\hline
name & VARCHAR(255) & UNIQUE, NOT NULL & Название миграции \\
\hline
checksum & CHAR(64) & NULL & SHA-256 hash \\
\hline
applied\_at & TIMESTAMPTZ & DEFAULT NOW() & Дата применения \\
\hline
execution\_time\_ms & INTEGER & NULL & Время выполнения \\
\hline
status & VARCHAR(20) & DEFAULT 'success', CHECK IN (...) & Статус \\
\hline
applied\_by & TEXT & NULL & Кто применил \\
\hline
\end{longtable}

Уникальные типы: UUID, VARCHAR, CHAR, TIMESTAMPTZ, INTEGER, TEXT = \textbf{6 типов}.

\subsection{Ограничения целостности данных}

\subsubsection{PRIMARY KEY}

Все таблицы имеют PRIMARY KEY на поле \code{id} типа UUID с автоматической генерацией значения:

\begin{lstlisting}
id UUID PRIMARY KEY DEFAULT gen_random_uuid()
\end{lstlisting}

\subsubsection{FOREIGN KEY}

Внешние ключи обеспечивают ссылочную целостность данных. Примеры:

\begin{lstlisting}
-- Связь habits с users
FOREIGN KEY (user_id) REFERENCES users(id) ON DELETE CASCADE

-- Связь habit_checkins с habits
FOREIGN KEY (habit_id) REFERENCES habits(id) ON DELETE CASCADE

-- Связь habit_tags с habits и tags
FOREIGN KEY (habit_id) REFERENCES habits(id) ON DELETE CASCADE
FOREIGN KEY (tag_id) REFERENCES tags(id) ON DELETE CASCADE
\end{lstlisting}

Каскадное удаление (ON DELETE CASCADE) применяется для зависимых записей, которые не имеют смысла без родительской записи. Для опциональных связей используется ON DELETE SET NULL.

\subsubsection{UNIQUE}

Уникальные ограничения обеспечивают неповторимость значений:

\begin{itemize}
    \item \code{users.email} --- уникальный email;
    \item \code{user\_profiles.user\_id} --- уникальная связь 1:1;
    \item \code{habit\_checkins(habit\_id, checkin\_date)} --- уникальная отметка на день;
    \item \code{tags.name} и \code{tags.slug} --- уникальные названия и слоги тегов;
    \item \code{\_manual\_migrations.name} --- уникальное имя миграции.
\end{itemize}

\subsubsection{CHECK}

Проверочные ограничения обеспечивают валидность данных:

\begin{lstlisting}
-- Тип привычки
CHECK (type IN ('good', 'bad'))

-- Оценка настроения
CHECK (mood_rating >= 1 AND mood_rating <= 5)

-- Приоритет
CHECK (priority >= 0 AND priority <= 10)

-- Операция аудита
CHECK (operation IN ('INSERT', 'UPDATE', 'DELETE'))

-- Статус задачи импорта
CHECK (status IN ('pending', 'processing', 'completed', 'failed'))
\end{lstlisting}

\subsubsection{NOT NULL}

Обязательные поля обеспечивают полноту данных:

\begin{itemize}
    \item \code{users.email}, \code{users.password\_hash}, \code{users.full\_name};
    \item \code{habits.name}, \code{habits.type}, \code{habits.user\_id};
    \item \code{habit\_checkins.habit\_id}, \code{habit\_checkins.user\_id}, \code{habit\_checkins.checkin\_date};
    \item \code{audit\_log.table\_name}, \code{audit\_log.operation}.
\end{itemize}

\subsection{Реализация SQL-функциональности}

\subsubsection{CRUD-операции}

Базовые CRUD-операции реализованы через Prisma ORM, что обеспечивает типобезопасность и автоматическую параметризацию запросов. Пример создания привычки:

\begin{lstlisting}[language=TypeScript]
async create(userId: string, dto: CreateHabitDto) {
  return await this.prisma.habit.create({
    data: {
      userId,
      name: dto.name,
      type: dto.type,
      priority: dto.priority,
    },
  });
}
\end{lstlisting}

Сложные запросы с агрегацией и JOIN выполняются через параметризованный Raw SQL:

\begin{lstlisting}[language=TypeScript]
const result = await this.prisma.$queryRaw`
  SELECT * FROM report_user_habits(
    ${userId}::uuid,
    ${from}::date,
    ${to}::date
  )
`;
\end{lstlisting}

Все SQL-запросы параметризованы для защиты от SQL-инъекций.

\subsubsection{Триггеры}

\textbf{Триггер аудита изменений}

Функция \code{audit\_trigger\_fn()} записывает в журнал аудита все изменения в ключевых таблицах:

\begin{lstlisting}
CREATE OR REPLACE FUNCTION audit_trigger_fn() RETURNS TRIGGER AS $$
DECLARE
  v_user_id UUID;
BEGIN
  -- Получаем user_id из контекста сессии
  BEGIN
    v_user_id := nullif(current_setting('app.user_id', true), '')::UUID;
  EXCEPTION WHEN OTHERS THEN
    v_user_id := NULL;
  END;

  -- Записываем аудит
  INSERT INTO audit_log (table_name, operation, record_id, user_id, 
                         old_data, new_data, changed_at)
  VALUES (
    TG_TABLE_NAME::VARCHAR,
    TG_OP::VARCHAR,
    COALESCE(NEW.id, OLD.id),
    v_user_id,
    CASE WHEN TG_OP IN ('UPDATE', 'DELETE') 
         THEN row_to_json(OLD)::jsonb ELSE NULL END,
    CASE WHEN TG_OP IN ('INSERT', 'UPDATE') 
         THEN row_to_json(NEW)::jsonb ELSE NULL END,
    NOW()
  );

  IF TG_OP = 'DELETE' THEN
    RETURN OLD;
  ELSE
    RETURN NEW;
  END IF;
END;
$$ LANGUAGE plpgsql;
\end{lstlisting}

Триггеры установлены на таблицы: \code{habits}, \code{habit\_checkins}, \code{tags}, \code{reminders}.

\textbf{Триггер автоматического обновления агрегатов}

Функция \code{update\_habit\_stats()} автоматически обновляет статистику привычки при изменении отметок:

\begin{lstlisting}
CREATE OR REPLACE FUNCTION update_habit_stats() RETURNS TRIGGER AS $$
DECLARE
  v_total INTEGER;
  v_avg_mood NUMERIC;
  v_last_checkin DATE;
BEGIN
  IF TG_OP = 'INSERT' THEN
    -- Вычисляем агрегаты
    SELECT COUNT(*), AVG(mood_rating), MAX(checkin_date)
    INTO v_total, v_avg_mood, v_last_checkin
    FROM habit_checkins
    WHERE habit_id = NEW.habit_id;
    
    -- Обновляем habit_stats
    INSERT INTO habit_stats (habit_id, total_checkins, average_mood, 
                             last_checkin_at, updated_at)
    VALUES (NEW.habit_id, v_total, v_avg_mood, v_last_checkin, NOW())
    ON CONFLICT (habit_id) DO UPDATE SET
      total_checkins = v_total,
      average_mood = v_avg_mood,
      last_checkin_at = GREATEST(habit_stats.last_checkin_at, v_last_checkin),
      updated_at = NOW();
  ELSIF TG_OP = 'DELETE' THEN
    -- Пересчитываем при удалении
    SELECT COUNT(*), AVG(mood_rating), MAX(checkin_date)
    INTO v_total, v_avg_mood, v_last_checkin
    FROM habit_checkins
    WHERE habit_id = OLD.habit_id;
    
    UPDATE habit_stats SET
      total_checkins = v_total,
      average_mood = v_avg_mood,
      last_checkin_at = v_last_checkin,
      updated_at = NOW()
    WHERE habit_id = OLD.habit_id;
  END IF;
  
  RETURN COALESCE(NEW, OLD);
END;
$$ LANGUAGE plpgsql;
\end{lstlisting}

Триггер установлен на таблицу \code{habit\_checkins}.

\subsubsection{Функции}

\textbf{Скалярная функция \code{calc\_completion\_rate}}

Вычисляет процент выполнения привычек пользователя за период:

\begin{lstlisting}
CREATE OR REPLACE FUNCTION calc_completion_rate(
  p_user_id UUID,
  p_from DATE,
  p_to DATE
) RETURNS NUMERIC AS $$
DECLARE
  v_expected INTEGER;
  v_actual INTEGER;
BEGIN
  -- Ожидаемое кол-во (дни * активные привычки)
  SELECT COUNT(DISTINCT h.id) * (p_to - p_from + 1)
  INTO v_expected
  FROM habits h
  WHERE h.user_id = p_user_id AND NOT h.is_archived;
  
  -- Фактическое кол-во checkins
  SELECT COUNT(*)
  INTO v_actual
  FROM habit_checkins hc
  JOIN habits h ON h.id = hc.habit_id
  WHERE h.user_id = p_user_id
    AND hc.checkin_date BETWEEN p_from AND p_to;
  
  IF v_expected = 0 THEN
    RETURN 0;
  END IF;
  
  RETURN ROUND((v_actual::NUMERIC / v_expected) * 100, 2);
END;
$$ LANGUAGE plpgsql;
\end{lstlisting}

\textbf{Табличная функция \code{report\_user\_habits}}

Возвращает отчёт по привычкам пользователя за период:

\begin{lstlisting}
CREATE OR REPLACE FUNCTION report_user_habits(
  p_user_id UUID,
  p_from DATE,
  p_to DATE
) RETURNS TABLE(
  habit_id UUID,
  habit_name TEXT,
  habit_type VARCHAR,
  total_checkins BIGINT,
  completion_rate NUMERIC,
  last_checkin DATE
) AS $$
BEGIN
  RETURN QUERY
  SELECT 
    h.id,
    h.name,
    h.type,
    COUNT(hc.id) as total_checkins,
    ROUND(
      (COUNT(hc.id)::NUMERIC / NULLIF((p_to - p_from + 1), 0)) * 100, 
      2
    ) as completion_rate,
    MAX(hc.checkin_date) as last_checkin
  FROM habits h
  LEFT JOIN habit_checkins hc ON hc.habit_id = h.id 
    AND hc.checkin_date BETWEEN p_from AND p_to
  WHERE h.user_id = p_user_id
  GROUP BY h.id, h.name, h.type
  ORDER BY total_checkins DESC;
END;
$$ LANGUAGE plpgsql;
\end{lstlisting}

\subsubsection{Представления (VIEW)}

\textbf{VIEW 1: \code{v\_user\_habit\_summary}}

Сводка по привычкам пользователя с агрегацией:

\begin{lstlisting}
CREATE OR REPLACE VIEW v_user_habit_summary AS
SELECT 
  u.id as user_id,
  u.email,
  u.full_name,
  COUNT(DISTINCT h.id) as total_habits,
  COUNT(DISTINCT CASE WHEN h.type = 'good' THEN h.id END) as good_habits,
  COUNT(DISTINCT CASE WHEN h.type = 'bad' THEN h.id END) as bad_habits,
  COALESCE(SUM(hs.total_checkins), 0) as total_checkins,
  MAX(hs.last_checkin_at) as last_activity
FROM users u
LEFT JOIN habits h ON h.user_id = u.id AND NOT h.is_archived
LEFT JOIN habit_stats hs ON hs.habit_id = h.id
GROUP BY u.id, u.email, u.full_name;
\end{lstlisting}

\textbf{VIEW 2: \code{v\_daily\_completion}}

Ежедневная статистика выполнения привычек:

\begin{lstlisting}
CREATE OR REPLACE VIEW v_daily_completion AS
SELECT 
  hc.checkin_date,
  COUNT(*) as total_checkins,
  COUNT(DISTINCT hc.user_id) as active_users,
  COUNT(DISTINCT hc.habit_id) as habits_completed,
  ROUND(AVG(hc.mood_rating), 2) as avg_mood,
  SUM(hc.duration_minutes) as total_minutes
FROM habit_checkins hc
GROUP BY hc.checkin_date
ORDER BY hc.checkin_date DESC;
\end{lstlisting}

\textbf{VIEW 3: \code{v\_tag\_usage}}

Использование тегов в системе:

\begin{lstlisting}
CREATE OR REPLACE VIEW v_tag_usage AS
SELECT 
  t.id as tag_id,
  t.name as tag_name,
  t.color,
  COUNT(ht.habit_id) as usage_count,
  COUNT(DISTINCT ht.habit_id) as unique_habits,
  MAX(ht.assigned_at) as last_used
FROM tags t
LEFT JOIN habit_tags ht ON ht.tag_id = t.id
GROUP BY t.id, t.name, t.color
ORDER BY usage_count DESC;
\end{lstlisting}

\subsection{Описание API и взаимодействия с БД}

\subsubsection{Архитектура API}

REST API реализован на базе фреймворка NestJS и предоставляет следующие группы endpoints:

\begin{itemize}
    \item \textbf{Аутентификация:} регистрация, вход, получение информации о текущем пользователе;
    \item \textbf{Привычки:} CRUD-операции для управления привычками;
    \item \textbf{Отметки выполнения:} создание и получение отметок выполнения;
    \item \textbf{Отчёты:} аналитические отчеты с использованием SQL-функций;
    \item \textbf{Batch Import:} массовая загрузка данных с отслеживанием прогресса.
\end{itemize}

Аутентификация осуществляется через JWT-токены. Все защищенные endpoints требуют заголовок \code{Authorization: Bearer <token>}.

\subsubsection{Основные endpoints}

\textbf{Аутентификация:}
\begin{itemize}
    \item \code{POST /auth/register} --- регистрация нового пользователя;
    \item \code{POST /auth/login} --- вход в систему;
    \item \code{GET /auth/me} --- получение информации о текущем пользователе.
\end{itemize}

\textbf{Привычки (CRUD):}
\begin{itemize}
    \item \code{GET /habits} --- получение списка привычек текущего пользователя;
    \item \code{POST /habits} --- создание новой привычки;
    \item \code{GET /habits/:id} --- получение привычки по ID;
    \item \code{PUT /habits/:id} --- обновление привычки;
    \item \code{DELETE /habits/:id} --- удаление привычки.
\end{itemize}

\textbf{Отметки выполнения:}
\begin{itemize}
    \item \code{POST /checkins} --- создание отметки выполнения;
    \item \code{GET /checkins} --- получение списка отметок с фильтрацией;
    \item \code{DELETE /checkins/:id} --- удаление отметки.
\end{itemize}

\textbf{Отчёты:}
\begin{itemize}
    \item \code{GET /reports/user/:userId} --- отчёт по привычкам пользователя (использует функцию \code{report\_user\_habits});
    \item \code{GET /reports/completion-rate/:userId} --- процент выполнения (использует функцию \code{calc\_completion\_rate}).
\end{itemize}

\textbf{Batch Import:}
\begin{itemize}
    \item \code{POST /batch-import} --- создание задачи массового импорта;
    \item \code{GET /batch-import/:jobId} --- получение статуса задачи;
    \item \code{GET /batch-import/:jobId/errors} --- получение ошибок импорта.
\end{itemize}

Полная документация API доступна через Swagger UI по адресу \code{http://localhost:3001/api/docs}.

\subsubsection{Безопасность}

Все SQL-запросы параметризованы для защиты от SQL-инъекций. Используется Prisma ORM для базовых операций и \code{\$queryRaw} с параметрами для сложных запросов:

\begin{lstlisting}[language=TypeScript]
// Правильно: параметризованный запрос
await prisma.$queryRaw`
  SELECT * FROM habits WHERE user_id = ${userId}
`;

// Неправильно: конкатенация строк (НЕ используется)
await prisma.$executeRawUnsafe(
  `SELECT * FROM habits WHERE user_id = '${userId}'`
);
\end{lstlisting}

Секретные данные (JWT\_SECRET, DATABASE\_URL) хранятся в переменных окружения (\code{.env} файл), который исключен из системы контроля версий.

\subsection{Примеры SQL-запросов и анализ производительности}

\subsubsection{Примеры SQL-запросов}

\textbf{Запрос 1: Получение привычек пользователя}

\begin{lstlisting}
SELECT * FROM habits 
WHERE user_id = $1 AND NOT is_archived
ORDER BY display_order;
\end{lstlisting}

\textbf{Запрос 2: Отчёт по привычкам через функцию}

\begin{lstlisting}
SELECT * FROM report_user_habits($1, $2, $3);
\end{lstlisting}

\textbf{Запрос 3: Использование VIEW}

\begin{lstlisting}
SELECT * FROM v_user_habit_summary 
WHERE user_id = $1;
\end{lstlisting}

\subsubsection{Анализ производительности}

Для оптимизации запросов созданы индексы на часто используемых полях. Ниже приведены результаты анализа производительности с помощью \code{EXPLAIN ANALYZE}.

\textbf{Запрос: Отметки пользователя за период}

Запрос для получения отметок пользователя за последние 30 дней:

\begin{lstlisting}
SELECT 
  hc.id, hc.checkin_date, hc.checkin_time, hc.mood_rating, h.name
FROM habit_checkins hc
JOIN habits h ON h.id = hc.habit_id
WHERE hc.user_id = 'user-uuid'
  AND hc.checkin_date >= CURRENT_DATE - INTERVAL '30 days'
ORDER BY hc.checkin_date DESC
LIMIT 100;
\end{lstlisting}

\textbf{До создания индексов:}
\begin{itemize}
    \item Execution Time: 145.456 ms
    \item Метод доступа: Sequential Scan на \code{habit\_checkins}
    \item Просмотрено строк: 10000
    \item Отфильтровано строк: 8658
\end{itemize}

\textbf{После создания индексов:}
\begin{itemize}
    \item Execution Time: 2.789 ms
    \item Метод доступа: Index Scan с использованием \code{idx\_habit\_checkins\_user\_date}
    \item Просмотрено строк: ~100
    \item Улучшение: \textbf{52x быстрее}
\end{itemize}

Созданный индекс:
\begin{lstlisting}
CREATE INDEX idx_habit_checkins_user_date 
  ON habit_checkins(user_id, checkin_date DESC);
\end{lstlisting}

\textbf{Запрос: Журнал аудита}

Запрос для получения действий пользователя за последние 24 часа:

\begin{lstlisting}
SELECT * FROM audit_log
WHERE user_id = 'user-uuid'
  AND changed_at > NOW() - INTERVAL '24 hours'
ORDER BY changed_at DESC
LIMIT 50;
\end{lstlisting}

\textbf{До создания индексов:}
\begin{itemize}
    \item Execution Time: 89.456 ms
    \item Метод доступа: Sequential Scan
    \item Просмотрено строк: 5000
    \item Отфильтровано строк: 4958
\end{itemize}

\textbf{После создания индексов:}
\begin{itemize}
    \item Execution Time: 1.289 ms
    \item Метод доступа: Index Scan с использованием \code{idx\_audit\_log\_user\_time}
    \item Просмотрено строк: ~42
    \item Улучшение: \textbf{69x быстрее}
\end{itemize}

Созданный индекс:
\begin{lstlisting}
CREATE INDEX idx_audit_log_user_time 
  ON audit_log(user_id, changed_at DESC) 
  WHERE user_id IS NOT NULL;
\end{lstlisting}

\textbf{Запрос: VIEW сводка привычек}

Запрос к представлению \code{v\_user\_habit\_summary}:

\begin{lstlisting}
SELECT * FROM v_user_habit_summary
WHERE user_id = 'user-uuid';
\end{lstlisting}

\textbf{До создания индексов:}
\begin{itemize}
    \item Execution Time: 89.789 ms
    \item Метод доступа: Sequential Scan на всех таблицах
    \item Join method: Hash Join
\end{itemize}

\textbf{После создания индексов:}
\begin{itemize}
    \item Execution Time: 1.289 ms
    \item Метод доступа: Index Scan на всех таблицах
    \item Join method: Nested Loop
    \item Улучшение: \textbf{69x быстрее}
\end{itemize}

Созданный индекс:
\begin{lstlisting}
CREATE INDEX idx_habits_user_id 
  ON habits(user_id) 
  WHERE NOT is_archived;
\end{lstlisting}

\textbf{Сводная таблица результатов оптимизации}

\begin{longtable}{|p{5cm}|p{3cm}|p{3cm}|p{3cm}|}
\hline
\textbf{Запрос} & \textbf{До (ms)} & \textbf{После (ms)} & \textbf{Улучшение} \\
\hline
\endfirsthead
\hline
\textbf{Запрос} & \textbf{До (ms)} & \textbf{После (ms)} & \textbf{Улучшение} \\
\hline
\endhead
\hline
\endfoot
\hline
\endlastfoot
Отметки пользователя & 145.456 & 2.789 & 52x \\
\hline
Журнал аудита & 89.456 & 1.289 & 69x \\
\hline
VIEW сводка привычек & 89.789 & 1.289 & 69x \\
\hline
\textbf{Среднее} & \textbf{108.2} & \textbf{1.79} & \textbf{60x} \\
\hline
\end{longtable}

Всего создано 10 индексов для оптимизации различных запросов. Использованы partial indexes (с условием WHERE) для уменьшения размера индексов и ускорения поиска.

% Технологическая часть
\newpage
\section{ТЕХНОЛОГИЧЕСКАЯ ЧАСТЬ}

\subsection{Контейнеризация и развертывание}

Система контейнеризована с использованием Docker и Docker Compose. Файл \code{docker-compose.yml} определяет три сервиса:

\begin{itemize}
    \item \textbf{db} --- PostgreSQL 16 с настройками для production;
    \item \textbf{backend} --- NestJS приложение на порту 3001;
    \item \textbf{frontend} --- Next.js приложение на порту 3000.
\end{itemize}

\textbf{Инструкции по развертыванию:}

\begin{enumerate}
    \item Установить зависимости: \code{pnpm install}
    \item Создать файл \code{.env} с необходимыми переменными окружения
    \item Запустить PostgreSQL: \code{docker-compose up -d db}
    \item Применить Prisma миграции: \code{pnpm db:migrate}
    \item Применить SQL-скрипты (триггеры, функции, views, индексы): \code{pnpm db:sql}
    \item Заполнить БД тестовыми данными: \code{pnpm db:seed}
    \item Запустить backend: \code{pnpm backend:dev}
    \item Запустить frontend: \code{pnpm frontend:dev}
\end{enumerate}

Для полного развертывания всех сервисов используется команда \code{docker-compose up -d}.

\subsection{Тестирование системы}

Система протестирована с использованием тестовых данных:
\begin{itemize}
    \item 100 пользователей;
    \item 800 привычек;
    \item 10000 отметок выполнения;
    \item 50 тегов;
    \item 1500 связей привычка-тег;
    \item 600 напоминаний.
\end{itemize}

Проверена работа триггеров аудита: при каждом изменении в таблицах \code{habits}, \code{habit\_checkins}, \code{tags}, \code{reminders} создается запись в \code{audit\_log} с полной информацией об изменении.

Проверена работа триггера агрегатов: при добавлении или удалении отметки автоматически обновляется статистика в \code{habit\_stats}.

Проверена работа функций: \code{calc\_completion\_rate} и \code{report\_user\_habits} возвращают корректные результаты.

Проверена работа представлений: все три VIEW возвращают корректные агрегированные данные.

Проверена работа batch import: массовая загрузка данных работает корректно, ошибки логируются в \code{batch\_import\_errors}, прогресс отслеживается в \code{batch\_import\_jobs}.

% Заключение
\newpage
\section{ЗАКЛЮЧЕНИЕ}

В ходе выполнения курсовой работы была создана полноценная информационная система для трекинга привычек с реляционной базой данных PostgreSQL.

\textbf{Основные результаты работы:}

\begin{itemize}
    \item Спроектирована и реализована база данных, содержащая 13 таблиц с различными типами связей (1:1, 1:N, N:M). Каждая таблица содержит не менее 5 различных типов данных.
    \item Реализованы все необходимые ограничения целостности: PRIMARY KEY, FOREIGN KEY с каскадным обновлением и удалением, UNIQUE, CHECK, NOT NULL.
    \item База данных наполнена реалистичными тестовыми данными: 100 пользователей, 800 привычек, 10000 отметок выполнения.
    \item Реализованы CRUD-операции через REST API: базовые операции через Prisma ORM, сложные запросы через параметризованный Raw SQL.
    \item Созданы триггеры для аудита изменений и автоматического обновления агрегированных данных.
    \item Реализованы скалярная функция \code{calc\_completion\_rate} и табличная функция \code{report\_user\_habits} для аналитики.
    \item Созданы три представления (VIEW) для агрегированных данных и аналитических отчетов.
    \item Оптимизированы запросы с помощью индексов: достигнуто улучшение производительности в среднем в 60 раз.
    \item Реализована батчевая загрузка данных с логированием ошибок и отслеживанием прогресса.
    \item Интегрирована база данных с backend-приложением на NestJS с полной документацией API через Swagger/OpenAPI.
    \item Система контейнеризована с использованием Docker и Docker Compose.
\end{itemize}

\textbf{Соответствие поставленным задачам:}

Все задачи, поставленные во введении, выполнены полностью. Система соответствует всем требованиям технического задания курсовой работы.

\textbf{Новизна и значимость:}

Работа демонстрирует навыки проектирования и реализации реляционных баз данных, использования современных технологий (PostgreSQL, NestJS, Prisma), оптимизации производительности запросов и обеспечения безопасности данных.

\textbf{Рекомендации по дальнейшему развитию:}

\begin{itemize}
    \item Расширение функциональности: добавление социальных функций (обмен привычками между пользователями), интеграция с календарями.
    \item Оптимизация производительности: использование материализованных представлений для сложных аналитических запросов, партиционирование больших таблиц.
    \item Улучшение аналитики: добавление машинного обучения для предсказания успешности выполнения привычек, рекомендательных систем.
    \item Масштабирование: использование read replicas для распределения нагрузки, кэширование часто запрашиваемых данных.
\end{itemize}

% Список использованных источников
\newpage
\begin{thebibliography}{9}

\bibitem{gost}
ГОСТ 7.32-2017. Отчет о научно-исследовательской работе. Структура и правила оформления. --- Введ. 2018-07-01. --- М.: Стандартинформ, 2018. --- 15 с.

\bibitem{postgresql}
PostgreSQL 16 Documentation [Электронный ресурс]. --- Режим доступа: https://www.postgresql.org/docs/16/ (дата обращения: 15.01.2025).

\bibitem{nestjs}
NestJS Documentation [Электронный ресурс]. --- Режим доступа: https://docs.nestjs.com/ (дата обращения: 15.01.2025).

\bibitem{prisma}
Prisma Documentation [Электронный ресурс]. --- Режим доступа: https://www.prisma.io/docs (дата обращения: 15.01.2025).

\bibitem{nextjs}
Next.js Documentation [Электронный ресурс]. --- Режим доступа: https://nextjs.org/docs (дата обращения: 15.01.2025).

\bibitem{docker}
Docker Documentation [Электронный ресурс]. --- Режим доступа: https://docs.docker.com/ (дата обращения: 15.01.2025).

\bibitem{silberschatz}
Силбершац А. Базы данных: проектирование, реализация и сопровождение. Теория и практика / А. Силбершац, Г. Корт, С. Сударашан. --- 3-е изд. --- М.: Вильямс, 2018. --- 1440 с.

\bibitem{date}
Дейт К. Дж. Введение в системы баз данных / К. Дж. Дейт. --- 8-е изд. --- М.: Вильямс, 2016. --- 1328 с.

\bibitem{connolly}
Коннолли Т. Базы данных. Проектирование, реализация и сопровождение. Теория и практика / Т. Коннолли, К. Бегг. --- 3-е изд. --- М.: Вильямс, 2017. --- 1440 с.

\end{thebibliography}

% Приложения
\newpage
\section*{ПРИЛОЖЕНИЕ А}
\addcontentsline{toc}{section}{ПРИЛОЖЕНИЕ А}
\section*{ER-диаграмма базы данных}

База данных содержит 13 таблиц со следующими связями:

\begin{itemize}
    \item \textbf{1:1:} users $\leftrightarrow$ user\_profiles, habits $\leftrightarrow$ habit\_stats
    \item \textbf{1:N:} users $\rightarrow$ habits, users $\rightarrow$ habit\_checkins, habits $\rightarrow$ habit\_checkins, habits $\rightarrow$ habit\_schedules, habits $\rightarrow$ reminders
    \item \textbf{N:M:} habits $\leftrightarrow$ tags (через habit\_tags)
\end{itemize}

\begin{figure}[H]
\centering
\resizebox{\textwidth}{!}{%
\begin{tikzpicture}[
    entity/.style={rectangle, draw=black, thick, minimum width=2cm, minimum height=0.8cm, align=center, font=\scriptsize},
    every node/.style={font=\scriptsize}
]

% Верхний ряд
\node[entity] (users) at (0,10) {users};
\node[entity] (user_profiles) at (-4,10) {user\_profiles};
\node[entity] (habits) at (4,10) {habits};

% Второй ряд
\node[entity] (habit_stats) at (6,8) {habit\_stats};
\node[entity] (habit_checkins) at (0,8) {habit\_checkins};
\node[entity] (audit_log) at (-6,8) {audit\_log};

% Третий ряд
\node[entity] (habit_schedules) at (4,6) {habit\_schedules};
\node[entity] (reminders) at (6,6) {reminders};
\node[entity] (tags) at (-4,6) {tags};
\node[entity] (batch_jobs) at (-6,6) {batch\_import\_jobs};

% Четвертый ряд
\node[entity] (habit_tags) at (0,4) {habit\_tags};
\node[entity] (batch_errors) at (-4,4) {batch\_import\_errors};
\node[entity] (migrations) at (6,4) {\_manual\_migrations};

% Связи 1:1
\draw[thick] (users) -- (user_profiles) node[midway, above, font=\tiny] {1:1};
\draw[thick] (habits) -- (habit_stats) node[midway, above, font=\tiny] {1:1};

% Связи 1:N от users
\draw[thick,->] (users) -- (habits) node[midway, above, font=\tiny] {1:N};
\draw[thick,->] (users) -- (habit_checkins) node[midway, left, font=\tiny] {1:N};
\draw[thick,->] (users) -- (batch_jobs) node[midway, above, font=\tiny] {1:N};

% Связи 1:N от habits
\draw[thick,->] (habits) -- (habit_checkins) node[midway, right, font=\tiny] {1:N};
\draw[thick,->] (habits) -- (habit_schedules) node[midway, right, font=\tiny] {1:N};
\draw[thick,->] (habits) -- (reminders) node[midway, above, font=\tiny] {1:N};
\draw[thick,->] (habits) -- (habit_tags) node[midway, right, font=\tiny] {1:N};

% Связи N:M через habit_tags
\draw[thick,<->] (tags) -- (habit_tags) node[midway, left, font=\tiny] {N:M};

% Связи от batch_jobs
\draw[thick,->] (batch_jobs) -- (batch_errors) node[midway, left, font=\tiny] {1:N};

\end{tikzpicture}%
}
\caption{ER-диаграмма базы данных}
\label{fig:erd}
\end{figure}

На рисунке \ref{fig:erd} представлена ER-диаграмма базы данных, показывающая все 13 таблиц и связи между ними. Основные сущности системы:
\begin{itemize}
    \item \textbf{users} --- центральная сущность, связанная с большинством других таблиц;
    \item \textbf{habits} --- основная бизнес-сущность, связанная с пользователями и множеством связанных сущностей;
    \item \textbf{habit\_checkins} --- транзакционная таблица с большим объемом данных;
    \item \textbf{tags} и \textbf{habit\_tags} --- реализация связи многие-ко-многим между привычками и тегами;
    \item \textbf{audit\_log} --- журнал аудита, независимая таблица для отслеживания изменений;
    \item \textbf{batch\_import\_jobs} и \textbf{batch\_import\_errors} --- таблицы для батчевого импорта данных.
\end{itemize}

\newpage
\section*{ПРИЛОЖЕНИЕ Б}
\addcontentsline{toc}{section}{ПРИЛОЖЕНИЕ Б}
\section*{Примеры SQL-запросов}

\subsection*{Б.1 Триггеры}

Полный код триггеров находится в файле \code{packages/db/sql/001\_triggers.sql}.

\subsection*{Б.2 Функции}

Полный код функций находится в файле \code{packages/db/sql/002\_functions.sql}.

\subsection*{Б.3 Представления (VIEW)}

Полный код представлений находится в файле \code{packages/db/sql/003\_views.sql}.

\subsection*{Б.4 Индексы}

Полный код индексов находится в файле \code{packages/db/sql/004\_indexes.sql}.

\newpage
\section*{ПРИЛОЖЕНИЕ В}
\addcontentsline{toc}{section}{ПРИЛОЖЕНИЕ В}
\section*{Swagger документация API}

Полная интерактивная документация API доступна через Swagger UI по адресу:

\code{http://localhost:3001/api/docs}

Swagger UI предоставляет возможность:
\begin{itemize}
    \item просмотра всех endpoints;
    \item тестирования запросов;
    \item просмотра схем данных;
    \item авторизации через JWT токен.
\end{itemize}

\newpage
\section*{ПРИЛОЖЕНИЕ Г}
\addcontentsline{toc}{section}{ПРИЛОЖЕНИЕ Г}
\section*{Результаты EXPLAIN ANALYZE}

Полные результаты анализа производительности запросов с помощью \code{EXPLAIN ANALYZE} находятся в файле \code{docs/perf.md}.

Основные результаты:
\begin{itemize}
    \item Запрос отметок пользователя: улучшение в 52 раза
    \item Запрос журнала аудита: улучшение в 69 раз
    \item Запрос VIEW сводка привычек: улучшение в 69 раз
    \item Среднее улучшение: 60 раз
\end{itemize}

\end{document}

