\documentclass[a4paper,14pt]{extarticle}

\usepackage[utf8]{inputenc}
\usepackage[T2A]{fontenc}
\usepackage{tempora}
\usepackage[russian]{babel}
\usepackage{cmap}
\usepackage{geometry}
\usepackage{setspace}
\usepackage{indentfirst}
\usepackage{enumitem}
\usepackage{graphicx}
\usepackage{float}
\usepackage{array}
\usepackage{tabularx}
\usepackage{booktabs}
\usepackage{longtable}
\usepackage{xcolor}
\usepackage{listings}
\usepackage{caption}
\usepackage[hidelinks]{hyperref}
\usepackage{tikz}
\usetikzlibrary{positioning,arrows.meta,shapes.multipart,calc,shadows}
\usepackage{titlesec}
\usepackage{fancyhdr}

\geometry{
  left=30mm,
  right=15mm,
  top=20mm,
  bottom=20mm
}

\onehalfspacing
\setlength{\parindent}{1.25cm}
\setlength{\parskip}{0pt}
\sloppy

\setlist[itemize]{label=--, leftmargin=1.25cm, itemsep=0pt, topsep=0pt}
\setlist[enumerate]{label=\arabic*), leftmargin=1.25cm, itemsep=0pt, topsep=0pt}

\titleformat{\section}{\normalfont\Large}{\thesection}{1em}{}
\titleformat{\subsection}{\normalfont\large}{\thesubsection}{1em}{}
\titleformat{\subsubsection}{\normalfont\normalsize}{\thesubsubsection}{1em}{}

\DeclareCaptionFont{singlespacing}{\setstretch{1}}
\captionsetup{labelsep=emdash, font=singlespacing}

\renewcommand{\contentsname}{СОДЕРЖАНИЕ}
\renewcommand{\lstlistingname}{Листинг}

\lstset{
  language=SQL,
  basicstyle=\ttfamily\small,
  keywordstyle=\color{blue},
  commentstyle=\color{gray},
  stringstyle=\color{teal},
  numbers=left,
  numberstyle=\tiny,
  numbersep=6pt,
  stepnumber=1,
  frame=single,
  breaklines=true,
  showstringspaces=false,
  columns=fullflexible,
  tabsize=2
}

\pagestyle{fancy}
\fancyhf{}
\fancyfoot[C]{\thepage}
\renewcommand{\headrulewidth}{0pt}

\tikzset{
  block/.style={draw, rounded corners, align=center, minimum height=1.1cm, minimum width=3.4cm},
  arrow/.style={-{Latex[length=3mm]}, thick}
}

\begin{document}
\pagenumbering{arabic}

\begin{titlepage}
\thispagestyle{empty}
\begin{singlespace}
\begin{center}
Министерство науки и высшего образования Российской Федерации \\
Федеральное государственное бюджетное образовательное учреждение высшего образования \\
«Московский авиационный институт (национальный исследовательский университет)» \\
Институт № 8 «Компьютерные науки и прикладная математика» \\
Кафедра 806 «Вычислительная математика и программирование»
\end{center}

\vspace{30mm}

\begin{center}
КУРСОВОЙ ПРОЕКТ \\
по дисциплине «Базы данных» \\
на тему: «Информационная система для трекинга привычек»
\end{center}

\vfill

\begin{flushright}
Выполнил: студент группы М8О-309Б-23 \\
Чесноков В.\,Д. \\
Руководитель: Грубенко М.\,Д.
\end{flushright}

\vfill

\begin{center}
Москва 2025
\end{center}
\end{singlespace}
\end{titlepage}

\newpage
\section*{СПИСОК ИСПОЛНИТЕЛЕЙ}
\addcontentsline{toc}{section}{СПИСОК ИСПОЛНИТЕЛЕЙ}
\begin{singlespace}
Студент группы М8О-309Б-23 Чесноков В.\,Д. --- исполнитель. \\
Подпись: \underline{\hspace{3cm}} \hfill Дата: \underline{\hspace{2cm}} 2025~г.

\vspace{5mm}

Руководитель: Грубенко М.\,Д. --- руководитель работы. \\
Подпись: \underline{\hspace{3cm}} \hfill Дата: \underline{\hspace{2cm}} 2025~г.
\end{singlespace}

\newpage
\section*{РЕФЕРАТ}
\addcontentsline{toc}{section}{РЕФЕРАТ}
Отчет 30~с., 1~кн., 3~рис., 3~табл., 7~источн., 4~прил.

КЛЮЧЕВЫЕ СЛОВА: база данных, PostgreSQL, триггеры, функции, представления, аудит, индексы, REST API, Swagger, контейнеризация.

В отчете описана разработка информационной системы для трекинга привычек с реляционной базой данных PostgreSQL. Объект исследования --- процессы хранения и анализа данных о привычках пользователей. Цель работы --- спроектировать и реализовать базу данных и API, поддерживающие учет привычек, отметок выполнения и аналитических отчетов. Метод исследования --- проектирование реляционной схемы, нормализация, разработка ограничений целостности, индексов и экспериментальный анализ производительности с EXPLAIN ANALYZE. Полученные результаты: создана схема из 13 таблиц, реализованы триггеры аудита и агрегатов, функции и представления, batch import и REST API. Эффективность подтверждена ускорением ключевых запросов в десятки раз.

\newpage
\tableofcontents

\newpage
\section*{ВВЕДЕНИЕ}
\addcontentsline{toc}{section}{ВВЕДЕНИЕ}
Современные сервисы персональной продуктивности требуют надежного хранения данных, оперативной аналитики и прозрачного контроля изменений. Поэтому актуально проектирование информационных систем с реляционной базой данных, поддерживающей проверку целостности, аудит и оптимизацию запросов. Требования к структуре пояснительной записки определены ГОСТ 7.32-2017 \cite{gost732} и техническим заданием \cite{courseTask}.

Объект исследования --- информационная система для трекинга привычек пользователей. Предмет исследования --- проектирование и реализация реляционной базы данных и API, обеспечивающих хранение, обработку и анализ данных о привычках.

Цель работы --- разработать информационную систему с реляционной БД и серверным API, поддерживающими учет привычек, отметок выполнения и аналитические отчеты.

Для достижения цели решены следующие задачи:
\begin{enumerate}
  \item провести анализ предметной области и требований к системе;
  \item спроектировать структуру БД с необходимыми связями и типами данных;
  \item реализовать ограничения целостности (PRIMARY KEY, FOREIGN KEY, UNIQUE, CHECK, NOT NULL);
  \item разработать SQL-объекты: триггеры, функции и представления;
  \item реализовать REST API и механизм batch import;
  \item оптимизировать запросы индексами и выполнить анализ производительности;
  \item описать контейнеризацию, развертывание и тестирование.
\end{enumerate}

Практическая значимость работы заключается в создании полноценной БД и API, готовых к использованию в сервисе трекинга привычек. Структура отчета включает введение, аналитическую, проектную и технологическую части, заключение и приложения.

\newpage
\section{АНАЛИТИЧЕСКАЯ ЧАСТЬ}
\subsection{Обзор предметной области}
Сервис трекинга привычек предназначен для ведения списка привычек, контроля регулярности выполнения и получения аналитических отчетов. Для корректной работы системы требуется хранить учетные записи пользователей, перечни привычек, расписания, отметки выполнения, теги и напоминания, а также агрегированную статистику. Для контроля корректности и отслеживания изменений необходим журнал аудита.

Ключевые сущности предметной области:
\begin{itemize}
  \item пользователи и их профили;
  \item привычки и расписания выполнения;
  \item отметки выполнения и связанные параметры (настроение, длительность);
  \item теги и напоминания;
  \item агрегированная статистика и аудит изменений.
\end{itemize}

Для предметной области характерны частые транзакционные операции (ежедневные отметки) и аналитические выборки. Это обуславливает необходимость индексирования и оптимизации запросов.

\subsection{Постановка задачи}
В соответствии с техническим заданием \cite{courseTask} система должна содержать не менее 9--10 таблиц со связями 1:1, 1:N и N:M, при этом каждая таблица включает не менее пяти атрибутов различных типов. Требуется реализовать ограничения целостности, каскадные правила, триггеры аудита и агрегатов, SQL-функции (скалярные и табличные), а также не менее трех представлений. Для ключевых запросов необходимо создать индексы и показать ускорение с помощью EXPLAIN ANALYZE. API должно предоставлять CRUD-операции, отчеты, batch import и быть документировано через Swagger.

\newpage
\section{ПРОЕКТНАЯ ЧАСТЬ}
\subsection{Архитектура системы}
Система реализована как монорепозиторий с разделением на backend, frontend и пакет базы данных. Backend построен на NestJS \cite{nestjs} и взаимодействует с PostgreSQL 16 \cite{postgres} через ORM Prisma \cite{prisma}. Frontend реализован на Next.js \cite{nextjs}. Контейнеризация выполнена с помощью Docker и Docker Compose \cite{docker}. Архитектура взаимодействия компонентов представлена на рисунке \ref{fig:architecture}.

\begin{figure}[H]
\centering
\begin{tikzpicture}[node distance=2.2cm]
\node[block] (frontend) {Frontend\\Next.js};
\node[block, right=of frontend] (backend) {Backend API\\NestJS + Prisma};
\node[block, right=of backend] (db) {PostgreSQL 16\\База данных};
\draw[arrow] (frontend) -- (backend);
\draw[arrow] (backend) -- (db);
\end{tikzpicture}
\caption{Архитектура информационной системы}
\label{fig:architecture}
\end{figure}

\subsection{Проектирование структуры базы данных}
Схема БД включает 13 таблиц и удовлетворяет требованиям по числу сущностей, разнообразию типов данных и наличию связей разных типов. Используются типы данных UUID, TEXT, VARCHAR, CHAR, INTEGER, BIGINT, SMALLINT, NUMERIC, BOOLEAN, DATE, TIME, TIMESTAMPTZ, JSONB и INET. Основные связи:
\begin{itemize}
  \item \texttt{users} --- \texttt{user\_profiles} (1:1);
  \item \texttt{users} --- \texttt{habits} (1:N);
  \item \texttt{habits} --- \texttt{habit\_checkins} (1:N);
  \item \texttt{habits} --- \texttt{tags} (N:M через \texttt{habit\_tags});
  \item \texttt{habits} --- \texttt{habit\_stats} (1:1);
  \item \texttt{batch\_import\_jobs} --- \texttt{batch\_import\_errors} (1:N).
\end{itemize}

Логическая схема базы данных с основными сущностями и атрибутами представлена в приложении А (рисунок \ref{fig:db-schema}).

\subsection{Описание таблиц базы данных}
Ключевые таблицы и их назначение приведены в таблице \ref{tab:db-tables}. Полный перечень содержит 13 таблиц: \texttt{users}, \texttt{user\_profiles}, \texttt{habits}, \texttt{habit\_schedules}, \texttt{habit\_checkins}, \texttt{tags}, \texttt{habit\_tags}, \texttt{reminders}, \texttt{habit\_stats}, \texttt{audit\_log}, \texttt{\_manual\_migrations}, \texttt{batch\_import\_jobs}, \texttt{batch\_import\_errors}.

\begin{table}[H]
\caption{Основные таблицы базы данных}
\label{tab:db-tables}
\centering
\renewcommand{\arraystretch}{1.2}
\small
\begin{tabularx}{\textwidth}{>{\raggedright\arraybackslash}p{4.4cm} X >{\raggedright\arraybackslash}p{4.0cm}}
\toprule
Таблица & Назначение & Ключевые поля \\
\midrule
\texttt{users} & учетные записи пользователей & \texttt{id}, \texttt{email} \\
\texttt{user\_\allowbreak profiles} & профиль пользователя & \texttt{user\_id}, \texttt{timezone} \\
\texttt{habits} & перечень привычек & \texttt{user\_id}, \texttt{name}, \texttt{type} \\
\texttt{habit\_\allowbreak schedules} & расписания привычек & \texttt{habit\_id}, \texttt{frequency\_type} \\
\texttt{habit\_\allowbreak checkins} & отметки выполнения & \texttt{habit\_id}, \texttt{checkin\_date} \\
\texttt{tags} & справочник тегов & \texttt{name}, \texttt{slug} \\
\texttt{habit\_\allowbreak tags} & связь привычек и тегов & \texttt{habit\_id}, \texttt{tag\_id} \\
\texttt{reminders} & напоминания & \texttt{habit\_id}, \texttt{reminder\_time} \\
\texttt{habit\_\allowbreak stats} & агрегированная статистика & \texttt{habit\_id}, \texttt{total\_checkins} \\
\texttt{audit\_\allowbreak log} & журнал аудита & \texttt{table\_name}, \texttt{operation} \\
\texttt{batch\_\allowbreak import\_\allowbreak jobs} & задания импорта & \texttt{entity\_type}, \texttt{status} \\
\texttt{batch\_\allowbreak import\_\allowbreak errors} & ошибки импорта & \texttt{job\_id}, \texttt{error\_message} \\
\texttt{\_manual\_\allowbreak migrations} & учет SQL-миграций & \texttt{name}, \texttt{status} \\
\bottomrule
\end{tabularx}
\normalsize
\end{table}

\subsubsection{Таблица users}
Назначение: основная таблица пользователей системы.

\begin{longtable}{|p{3cm}|p{2.5cm}|p{4cm}|p{5cm}|}
\hline
\textbf{Колонка} & \textbf{Тип} & \textbf{Ограничения} & \textbf{Описание} \\
\hline
\endfirsthead
\hline
\textbf{Колонка} & \textbf{Тип} & \textbf{Ограничения} & \textbf{Описание} \\
\hline
\endhead
\hline
\endfoot
\hline
\endlastfoot
id & UUID & PRIMARY KEY, DEFAULT gen\_random\_uuid() & Уникальный идентификатор \\
\hline
email & TEXT & UNIQUE, NOT NULL & Email пользователя \\
\hline
password\_hash & TEXT & NOT NULL & Хеш пароля \\
\hline
full\_name & TEXT & NOT NULL & Полное имя \\
\hline
role & VARCHAR(50) & DEFAULT 'user' & Роль пользователя \\
\hline
is\_active & BOOLEAN & DEFAULT true & Активен ли аккаунт \\
\hline
created\_at & TIMESTAMPTZ & DEFAULT NOW() & Дата создания \\
\hline
login\_count & INTEGER & DEFAULT 0 & Количество входов \\
\hline
\end{longtable}

Уникальные типы: UUID, TEXT, VARCHAR, BOOLEAN, TIMESTAMPTZ, INTEGER.

\subsubsection{Таблица user\_profiles}
Назначение: дополнительная информация о пользователе (связь 1:1 с users).

\begin{longtable}{|p{3cm}|p{2.5cm}|p{4cm}|p{5cm}|}
\hline
\textbf{Колонка} & \textbf{Тип} & \textbf{Ограничения} & \textbf{Описание} \\
\hline
\endfirsthead
\hline
\textbf{Колонка} & \textbf{Тип} & \textbf{Ограничения} & \textbf{Описание} \\
\hline
\endhead
\hline
\endfoot
\hline
\endlastfoot
id & UUID & PRIMARY KEY, DEFAULT gen\_random\_uuid() & Уникальный идентификатор \\
\hline
user\_id & UUID & UNIQUE, NOT NULL, FK -> users & Ссылка на пользователя \\
\hline
bio & TEXT & NULL & Биография \\
\hline
avatar\_url & TEXT & NULL & URL аватара \\
\hline
timezone & VARCHAR(100) & DEFAULT 'UTC' & Часовой пояс \\
\hline
date\_of\_birth & DATE & NULL & Дата рождения \\
\hline
notification\_enabled & BOOLEAN & DEFAULT true & Уведомления включены \\
\hline
theme\_preference & SMALLINT & DEFAULT 0, CHECK (0--2) & Тема оформления \\
\hline
\end{longtable}

Уникальные типы: UUID, TEXT, VARCHAR, DATE, BOOLEAN, SMALLINT.

\subsubsection{Таблица habits}
Назначение: привычки пользователей (хорошие и плохие).

\begin{longtable}{|p{3cm}|p{2.5cm}|p{4cm}|p{5cm}|}
\hline
\textbf{Колонка} & \textbf{Тип} & \textbf{Ограничения} & \textbf{Описание} \\
\hline
\endfirsthead
\hline
\textbf{Колонка} & \textbf{Тип} & \textbf{Ограничения} & \textbf{Описание} \\
\hline
\endhead
\hline
\endfoot
\hline
\endlastfoot
id & UUID & PRIMARY KEY, DEFAULT gen\_random\_uuid() & Уникальный идентификатор \\
\hline
user\_id & UUID & NOT NULL, FK -> users & Владелец привычки \\
\hline
name & TEXT & NOT NULL & Название привычки \\
\hline
description & TEXT & NULL & Описание \\
\hline
type & VARCHAR(10) & NOT NULL, CHECK IN ('good', 'bad') & Тип привычки \\
\hline
priority & SMALLINT & DEFAULT 0, CHECK (0--10) & Приоритет \\
\hline
is\_archived & BOOLEAN & DEFAULT false & Архивирована \\
\hline
display\_order & INTEGER & DEFAULT 0 & Порядок отображения \\
\hline
created\_at & TIMESTAMPTZ & DEFAULT NOW() & Дата создания \\
\hline
\end{longtable}

Уникальные типы: UUID, TEXT, VARCHAR, SMALLINT, BOOLEAN, INTEGER, TIMESTAMPTZ.

\subsubsection{Таблица habit\_schedules}
Назначение: расписания для привычек (ежедневные, еженедельные и т.д.).

\begin{longtable}{|p{3cm}|p{2.5cm}|p{4cm}|p{5cm}|}
\hline
\textbf{Колонка} & \textbf{Тип} & \textbf{Ограничения} & \textbf{Описание} \\
\hline
\endfirsthead
\hline
\textbf{Колонка} & \textbf{Тип} & \textbf{Ограничения} & \textbf{Описание} \\
\hline
\endhead
\hline
\endfoot
\hline
\endlastfoot
id & UUID & PRIMARY KEY & Уникальный идентификатор \\
\hline
habit\_id & UUID & NOT NULL, FK -> habits & Ссылка на привычку \\
\hline
frequency\_type & VARCHAR(20) & CHECK IN (...) & Тип частоты \\
\hline
frequency\_value & INTEGER & DEFAULT 1, CHECK > 0 & Значение частоты \\
\hline
weekdays\_mask & SMALLINT & DEFAULT 127 & Битовая маска дней недели \\
\hline
start\_date & DATE & NOT NULL & Дата начала \\
\hline
end\_date & DATE & NULL & Дата окончания \\
\hline
is\_active & BOOLEAN & DEFAULT true & Активно \\
\hline
\end{longtable}

Уникальные типы: UUID, VARCHAR, INTEGER, SMALLINT, DATE, BOOLEAN.

\subsubsection{Таблица habit\_checkins}
Назначение: отметки выполнения привычек (транзакционная таблица).

\begin{longtable}{|p{3cm}|p{2.5cm}|p{4cm}|p{5cm}|}
\hline
\textbf{Колонка} & \textbf{Тип} & \textbf{Ограничения} & \textbf{Описание} \\
\hline
\endfirsthead
\hline
\textbf{Колонка} & \textbf{Тип} & \textbf{Ограничения} & \textbf{Описание} \\
\hline
\endhead
\hline
\endfoot
\hline
\endlastfoot
id & UUID & PRIMARY KEY & Уникальный идентификатор \\
\hline
habit\_id & UUID & NOT NULL, FK -> habits & Ссылка на привычку \\
\hline
user\_id & UUID & NOT NULL, FK -> users & Ссылка на пользователя \\
\hline
checkin\_date & DATE & NOT NULL & Дата отметки \\
\hline
checkin\_time & TIME & DEFAULT CURRENT\_TIME & Время отметки \\
\hline
notes & TEXT & NULL & Заметки \\
\hline
mood\_rating & SMALLINT & CHECK (1--5) & Оценка настроения \\
\hline
duration\_minutes & INTEGER & NULL & Длительность в минутах \\
\hline
created\_at & TIMESTAMPTZ & DEFAULT NOW() & Дата создания \\
\hline
\end{longtable}

Уникальные типы: UUID, DATE, TIME, TEXT, SMALLINT, INTEGER, TIMESTAMPTZ. UNIQUE: (habit\_id, checkin\_date).

\subsubsection{Таблица tags}
Назначение: теги для категоризации привычек.

\begin{longtable}{|p{3cm}|p{2.5cm}|p{4cm}|p{5cm}|}
\hline
\textbf{Колонка} & \textbf{Тип} & \textbf{Ограничения} & \textbf{Описание} \\
\hline
\endfirsthead
\hline
\textbf{Колонка} & \textbf{Тип} & \textbf{Ограничения} & \textbf{Описание} \\
\hline
\endhead
\hline
\endfoot
\hline
\endlastfoot
id & UUID & PRIMARY KEY & Уникальный идентификатор \\
\hline
name & VARCHAR(100) & UNIQUE, NOT NULL & Название тега \\
\hline
slug & TEXT & UNIQUE, NOT NULL & URL-friendly имя \\
\hline
color & VARCHAR(7) & DEFAULT '\#gray' & Цвет \\
\hline
usage\_count & INTEGER & DEFAULT 0 & Счётчик использований \\
\hline
is\_system & BOOLEAN & DEFAULT false & Системный тег \\
\hline
created\_at & TIMESTAMPTZ & DEFAULT NOW() & Дата создания \\
\hline
\end{longtable}

Уникальные типы: UUID, VARCHAR, TEXT, INTEGER, BOOLEAN, TIMESTAMPTZ.

\subsubsection{Таблица habit\_tags}
Назначение: связь N:M между привычками и тегами.

\begin{longtable}{|p{3cm}|p{2.5cm}|p{4cm}|p{5cm}|}
\hline
\textbf{Колонка} & \textbf{Тип} & \textbf{Ограничения} & \textbf{Описание} \\
\hline
\endfirsthead
\hline
\textbf{Колонка} & \textbf{Тип} & \textbf{Ограничения} & \textbf{Описание} \\
\hline
\endhead
\hline
\endfoot
\hline
\endlastfoot
id & UUID & PRIMARY KEY & Уникальный идентификатор \\
\hline
habit\_id & UUID & NOT NULL, FK -> habits & Ссылка на привычку \\
\hline
tag\_id & UUID & NOT NULL, FK -> tags & Ссылка на тег \\
\hline
priority & SMALLINT & DEFAULT 0 & Приоритет \\
\hline
is\_primary & BOOLEAN & DEFAULT false & Основной тег \\
\hline
assigned\_by & VARCHAR(50) & DEFAULT 'user' & Кто назначил \\
\hline
assigned\_at & TIMESTAMPTZ & DEFAULT NOW() & Дата назначения \\
\hline
\end{longtable}

Уникальные типы: UUID, SMALLINT, BOOLEAN, VARCHAR, TIMESTAMPTZ. UNIQUE: (habit\_id, tag\_id).

\subsubsection{Таблица reminders}
Назначение: напоминания для привычек.

\begin{longtable}{|p{3cm}|p{2.5cm}|p{4cm}|p{5cm}|}
\hline
\textbf{Колонка} & \textbf{Тип} & \textbf{Ограничения} & \textbf{Описание} \\
\hline
\endfirsthead
\hline
\textbf{Колонка} & \textbf{Тип} & \textbf{Ограничения} & \textbf{Описание} \\
\hline
\endhead
\hline
\endfoot
\hline
\endlastfoot
id & UUID & PRIMARY KEY & Уникальный идентификатор \\
\hline
habit\_id & UUID & NOT NULL, FK -> habits & Ссылка на привычку \\
\hline
reminder\_time & TIME & NOT NULL & Время напоминания \\
\hline
days\_of\_week & SMALLINT & DEFAULT 127 & Дни недели (битовая маска) \\
\hline
notification\_text & TEXT & NULL & Текст уведомления \\
\hline
delivery\_method & VARCHAR(20) & DEFAULT 'push', CHECK IN (...) & Метод доставки \\
\hline
is\_active & BOOLEAN & DEFAULT true & Активно \\
\hline
created\_at & TIMESTAMPTZ & DEFAULT NOW() & Дата создания \\
\hline
\end{longtable}

Уникальные типы: UUID, TIME, SMALLINT, TEXT, VARCHAR, BOOLEAN, TIMESTAMPTZ.

\subsubsection{Таблица habit\_stats}
Назначение: агрегированная статистика по привычкам (обновляется триггером).

\begin{longtable}{|p{3cm}|p{2.5cm}|p{4cm}|p{5cm}|}
\hline
\textbf{Колонка} & \textbf{Тип} & \textbf{Ограничения} & \textbf{Описание} \\
\hline
\endfirsthead
\hline
\textbf{Колонка} & \textbf{Тип} & \textbf{Ограничения} & \textbf{Описание} \\
\hline
\endhead
\hline
\endfoot
\hline
\endlastfoot
id & UUID & PRIMARY KEY & Уникальный идентификатор \\
\hline
habit\_id & UUID & UNIQUE, NOT NULL, FK -> habits & Ссылка на привычку \\
\hline
total\_checkins & INTEGER & DEFAULT 0 & Всего отметок \\
\hline
current\_streak & INTEGER & DEFAULT 0 & Текущая серия \\
\hline
longest\_streak & INTEGER & DEFAULT 0 & Лучшая серия \\
\hline
completion\_rate & NUMERIC(5,2) & DEFAULT 0.00 & Процент выполнения \\
\hline
average\_mood & NUMERIC(3,2) & NULL & Средняя оценка настроения \\
\hline
last\_checkin\_at & DATE & NULL & Последняя отметка \\
\hline
updated\_at & TIMESTAMPTZ & DEFAULT NOW() & Дата обновления \\
\hline
\end{longtable}

Уникальные типы: UUID, INTEGER, NUMERIC, DATE, TIMESTAMPTZ.

\subsubsection{Таблица audit\_log}
Назначение: журнал аудита изменений.

\begin{longtable}{|p{3cm}|p{2.5cm}|p{4cm}|p{5cm}|}
\hline
\textbf{Колонка} & \textbf{Тип} & \textbf{Ограничения} & \textbf{Описание} \\
\hline
\endfirsthead
\hline
\textbf{Колонка} & \textbf{Тип} & \textbf{Ограничения} & \textbf{Описание} \\
\hline
\endhead
\hline
\endfoot
\hline
\endlastfoot
id & UUID & PRIMARY KEY & Уникальный идентификатор \\
\hline
table\_name & VARCHAR(100) & NOT NULL & Название таблицы \\
\hline
operation & VARCHAR(10) & NOT NULL, CHECK IN (...) & Тип операции \\
\hline
record\_id & UUID & NULL & ID изменённой записи \\
\hline
user\_id & UUID & NULL & ID пользователя \\
\hline
old\_data & JSONB & NULL & Старые данные \\
\hline
new\_data & JSONB & NULL & Новые данные \\
\hline
ip\_address & INET & NULL & IP-адрес \\
\hline
changed\_at & TIMESTAMPTZ & DEFAULT NOW() & Дата изменения \\
\hline
\end{longtable}

Уникальные типы: UUID, VARCHAR, JSONB, INET, TIMESTAMPTZ.

\subsubsection{Таблица \_manual\_migrations}
Назначение: контроль применения SQL-миграций (триггеров, функций, представлений и индексов).

\begin{longtable}{|p{3cm}|p{2.5cm}|p{4cm}|p{5cm}|}
\hline
\textbf{Колонка} & \textbf{Тип} & \textbf{Ограничения} & \textbf{Описание} \\
\hline
\endfirsthead
\hline
\textbf{Колонка} & \textbf{Тип} & \textbf{Ограничения} & \textbf{Описание} \\
\hline
\endhead
\hline
\endfoot
\hline
\endlastfoot
id & UUID & PRIMARY KEY & Уникальный идентификатор \\
\hline
name & VARCHAR(255) & UNIQUE, NOT NULL & Название миграции \\
\hline
checksum & CHAR(64) & NULL & SHA-256 hash \\
\hline
applied\_at & TIMESTAMPTZ & DEFAULT NOW() & Дата применения \\
\hline
execution\_time\_ms & INTEGER & NULL & Время выполнения \\
\hline
status & VARCHAR(20) & DEFAULT 'success', CHECK IN (...) & Статус \\
\hline
applied\_by & TEXT & NULL & Кто применил \\
\hline
\end{longtable}

Уникальные типы: UUID, VARCHAR, CHAR, TIMESTAMPTZ, INTEGER, TEXT.

\subsubsection{Таблица batch\_import\_jobs}
Назначение: задачи батчевого импорта данных.

\begin{longtable}{|p{3cm}|p{2.5cm}|p{4cm}|p{5cm}|}
\hline
\textbf{Колонка} & \textbf{Тип} & \textbf{Ограничения} & \textbf{Описание} \\
\hline
\endfirsthead
\hline
\textbf{Колонка} & \textbf{Тип} & \textbf{Ограничения} & \textbf{Описание} \\
\hline
\endhead
\hline
\endfoot
\hline
\endlastfoot
id & UUID & PRIMARY KEY & Уникальный идентификатор \\
\hline
user\_id & UUID & FK -> users, ON DELETE SET NULL & Пользователь \\
\hline
entity\_type & VARCHAR(50) & NOT NULL & Тип сущности \\
\hline
status & VARCHAR(20) & CHECK IN (...) & Статус \\
\hline
total\_records & INTEGER & DEFAULT 0 & Всего записей \\
\hline
success\_count & INTEGER & DEFAULT 0 & Успешных \\
\hline
error\_count & INTEGER & DEFAULT 0 & Ошибок \\
\hline
progress\_percent & NUMERIC(5,2) & DEFAULT 0.00 & Прогресс \\
\hline
file\_size\_bytes & BIGINT & NULL & Размер файла \\
\hline
started\_at & TIMESTAMPTZ & DEFAULT NOW() & Время начала \\
\hline
completed\_at & TIMESTAMPTZ & NULL & Время завершения \\
\hline
\end{longtable}

Уникальные типы: UUID, VARCHAR, INTEGER, NUMERIC, BIGINT, TIMESTAMPTZ.

\subsubsection{Таблица batch\_import\_errors}
Назначение: ошибки батчевого импорта.

\begin{longtable}{|p{3cm}|p{2.5cm}|p{4cm}|p{5cm}|}
\hline
\textbf{Колонка} & \textbf{Тип} & \textbf{Ограничения} & \textbf{Описание} \\
\hline
\endfirsthead
\hline
\textbf{Колонка} & \textbf{Тип} & \textbf{Ограничения} & \textbf{Описание} \\
\hline
\endhead
\hline
\endfoot
\hline
\endlastfoot
id & UUID & PRIMARY KEY & Уникальный идентификатор \\
\hline
job\_id & UUID & NOT NULL, FK -> batch\_import\_jobs & Ссылка на задачу \\
\hline
row\_number & INTEGER & NULL & Номер строки \\
\hline
record\_data & JSONB & NOT NULL & Данные записи \\
\hline
error\_message & TEXT & NOT NULL & Сообщение об ошибке \\
\hline
error\_code & VARCHAR(50) & NULL & Код ошибки \\
\hline
created\_at & TIMESTAMPTZ & DEFAULT NOW() & Дата создания \\
\hline
\end{longtable}

Уникальные типы: UUID, INTEGER, JSONB, TEXT, VARCHAR, TIMESTAMPTZ.

\subsection{Ограничения целостности данных}
Для обеспечения целостности реализованы следующие типы ограничений: PRIMARY KEY на поле \texttt{id} во всех таблицах, FOREIGN KEY с каскадным удалением или установкой \texttt{NULL}, UNIQUE для уникальных атрибутов, CHECK для доменных ограничений и NOT NULL для обязательных полей. Примеры:
\begin{itemize}
  \item PRIMARY KEY: \texttt{id UUID PRIMARY KEY DEFAULT gen\_random\_uuid()};
  \item FOREIGN KEY: \texttt{habits.user\_id -> users.id ON DELETE CASCADE};
  \item UNIQUE: \texttt{habit\_checkins(habit\_id, checkin\_date)};
  \item CHECK: \texttt{habits.type IN ('good', 'bad')}, \texttt{habit\_checkins.mood\_rating BETWEEN 1 AND 5};
  \item NOT NULL: \texttt{users.email}, \texttt{habits.name}, \texttt{habits.type}.
\end{itemize}

\subsection{Реализация SQL-функциональности}
Базовые CRUD-операции реализованы через Prisma ORM, а сложные выборки и отчеты --- через параметризованные SQL-запросы. Реализованы триггеры аудита и автоматического обновления агрегированной статистики, скалярная функция \texttt{calc\_completion\_rate} и табличная функция \texttt{report\_user\_habits}. Полные скрипты приведены в приложении Б и в файлах \texttt{packages/db/sql/001\_triggers.sql}--\texttt{004\_indexes.sql}.

\subsection{Описание API и взаимодействия с БД}
Backend предоставляет REST API с документацией в Swagger. Основные группы эндпойнтов:
\begin{itemize}
  \item аутентификация: \texttt{/auth/register}, \texttt{/auth/login}, \texttt{/auth/me};
  \item управление привычками: \texttt{/habits} (GET, POST), \texttt{/habits/:id} (GET, PUT, DELETE);
  \item отметки выполнения: \texttt{/checkins} (GET, POST), \texttt{/checkins/:id} (DELETE);
  \item отчеты: \texttt{/reports/user/:userId}, \texttt{/reports/completion-rate/:userId};
  \item массовая загрузка: \texttt{/batch-import} и \texttt{/batch-import/:jobId}.
\end{itemize}

Безопасность обеспечивается JWT-аутентификацией, хешированием паролей (bcrypt) и параметризованными запросами к БД. SQL-инъекции исключаются благодаря использованию ORM и параметризации.

\subsection{Примеры SQL-запросов и анализ производительности}
Примеры типовых запросов приведены в листинге \ref{lst:sample-queries}.

\begin{lstlisting}[caption=Примеры SQL-запросов, label=lst:sample-queries]
SELECT * FROM habits
WHERE user_id = $1 AND NOT is_archived
ORDER BY display_order;

SELECT * FROM report_user_habits($1, $2, $3);

SELECT * FROM v_user_habit_summary
WHERE user_id = $1;
\end{lstlisting}

Для повышения производительности созданы индексы на полях, используемых в фильтрации и сортировке. В таблице \ref{tab:perf} приведено сравнение времени выполнения запросов до и после индексирования. Полные планы EXPLAIN ANALYZE приведены в приложении Г.

\begin{table}[H]
\caption{Сравнение производительности запросов}
\label{tab:perf}
\centering
\renewcommand{\arraystretch}{1.2}
\begin{tabular}{p{6.5cm} p{2.5cm} p{2.5cm} p{2.5cm}}
\toprule
Запрос & До, мс & После, мс & Ускорение \\
\midrule
История отметок пользователя за 30 дней & 145.456 & 2.789 & 52x \\
Журнал аудита за 24 часа & 89.456 & 1.289 & 69x \\
\bottomrule
\end{tabular}
\end{table}

Ключевые индексы: \texttt{idx\_habits\_user\_id}, \texttt{idx\_habit\_checkins\_user\_date}, \texttt{idx\_audit\_log\_user\_time} и др. (см. приложение Б).

\newpage
\section{ТЕХНОЛОГИЧЕСКАЯ ЧАСТЬ}
\subsection{Контейнеризация и развертывание}
Развертывание системы выполняется с помощью Docker Compose. Используются три сервиса: база данных PostgreSQL, backend и frontend. Конфигурация размещена в файле \texttt{docker-compose.yml} и описывает порты, переменные окружения и зависимости сервисов.

После запуска контейнеров выполняются миграции и применение SQL-скриптов: \texttt{pnpm db:migrate}, \texttt{pnpm db:sql}, \texttt{pnpm db:seed}. Swagger доступен по адресу \texttt{http://localhost:3001/api/docs}.

\subsection{Тестирование системы}
Для проверки работоспособности используются тестовые данные и контрольные запросы. Скрипт заполнения БД формирует реалистичный набор данных, что позволяет проверить триггеры, функции и производительность.

\begin{table}[H]
\caption{Объем тестовых данных после заполнения}
\label{tab:seed-data}
\centering
\renewcommand{\arraystretch}{1.2}
\begin{tabular}{p{7cm} r}
\toprule
Таблица & Количество записей \\
\midrule
\texttt{users} & 100 \\
\texttt{habits} & 800 \\
\texttt{habit\_checkins} & 10000 \\
\texttt{tags} & 50 \\
\texttt{habit\_tags} & 1500 \\
\texttt{reminders} & 600 \\
\bottomrule
\end{tabular}
\end{table}

Примеры проверочных запросов включают выборку последних записей аудита, расчет \texttt{calc\_completion\_rate} за период и получение ежедневной статистики из представления \texttt{v\_daily\_completion}.

\newpage
\section*{ЗАКЛЮЧЕНИЕ}
\addcontentsline{toc}{section}{ЗАКЛЮЧЕНИЕ}
В ходе выполнения курсового проекта разработана информационная система для трекинга привычек. Спроектирована и реализована реляционная база данных из 13 таблиц с поддержкой связей 1:1, 1:N и N:M, реализованы ограничения целостности, триггеры аудита и агрегатов, функции и представления. Создан REST API с документацией Swagger и механизмом batch import. Проведена оптимизация запросов индексами и выполнен анализ производительности, показавший ускорение в десятки раз.

Поставленные задачи выполнены полностью. Результаты демонстрируют корректность структуры БД, устойчивость работы и готовность системы к дальнейшему развитию, включая расширение аналитики и функциональности.

\newpage
\section*{СПИСОК ИСПОЛЬЗОВАННЫХ ИСТОЧНИКОВ}
\addcontentsline{toc}{section}{СПИСОК ИСПОЛЬЗОВАННЫХ ИСТОЧНИКОВ}
\begin{thebibliography}{7}
  \bibitem{gost732} ГОСТ 7.32-2017. Отчет о научно-исследовательской работе. Структура и правила оформления.
  \bibitem{courseTask} Техническое задание на курсовую работу по дисциплине «Базы данных». 2025.
  \bibitem{postgres} PostgreSQL 16 Documentation [Электронный ресурс]. --- URL: \url{https://www.postgresql.org/docs/16/} (дата обращения: 01.03.2025).
  \bibitem{nestjs} NestJS Documentation [Электронный ресурс]. --- URL: \url{https://docs.nestjs.com/} (дата обращения: 01.03.2025).
  \bibitem{prisma} Prisma Documentation [Электронный ресурс]. --- URL: \url{https://www.prisma.io/docs} (дата обращения: 01.03.2025).
  \bibitem{nextjs} Next.js Documentation [Электронный ресурс]. --- URL: \url{https://nextjs.org/docs} (дата обращения: 01.03.2025).
  \bibitem{docker} Docker Documentation [Электронный ресурс]. --- URL: \url{https://docs.docker.com/} (дата обращения: 01.03.2025).
\end{thebibliography}

\newpage
\section*{ПРИЛОЖЕНИЕ А Логическая схема базы данных}
\addcontentsline{toc}{section}{ПРИЛОЖЕНИЕ А Логическая схема базы данных}
\setcounter{figure}{0}
\renewcommand{\thefigure}{А.\arabic{figure}}

\begin{figure}[H]
\centering
\resizebox{\textwidth}{!}{%
\begin{tikzpicture}[
    node distance=1.7cm and 2.2cm,
    table/.style={
        rectangle split, rectangle split parts=2,
        draw, fill=gray!5, rounded corners,
        minimum width=3.2cm, font=\scriptsize\ttfamily,
        align=left
    },
    edge/.style={->, >=Latex, thick}
]
    \node[table] (profiles) {
        user\_profiles
        \nodepart{second}
        \underline{id: UUID} \\
        user\_id: UUID (FK) \\
        bio: TEXT \\
        timezone: VARCHAR
    };

    \node[table, right=of profiles] (users) {
        users
        \nodepart{second}
        \underline{id: UUID} \\
        email: TEXT \\
        password\_hash: TEXT \\
        role: VARCHAR
    };

    \node[table, right=of users] (habits) {
        habits
        \nodepart{second}
        \underline{id: UUID} \\
        user\_id: UUID (FK) \\
        name: TEXT \\
        type: VARCHAR
    };

    \node[table, right=of habits] (tags) {
        tags
        \nodepart{second}
        \underline{id: UUID} \\
        name: VARCHAR \\
        slug: TEXT \\
        color: VARCHAR
    };

    \node[table, below=of users] (checkins) {
        habit\_checkins
        \nodepart{second}
        \underline{id: UUID} \\
        habit\_id: UUID (FK) \\
        user\_id: UUID (FK) \\
        checkin\_date: DATE
    };

    \node[table, below=of habits] (schedules) {
        habit\_schedules
        \nodepart{second}
        \underline{id: UUID} \\
        habit\_id: UUID (FK) \\
        frequency\_type: VARCHAR \\
        start\_date: DATE
    };

    \node[table, below=of tags] (habit_tags) {
        habit\_tags
        \nodepart{second}
        \underline{id: UUID} \\
        habit\_id: UUID (FK) \\
        tag\_id: UUID (FK) \\
        assigned\_at: TIMESTAMPTZ
    };

    \node[table, below=of profiles] (audit) {
        audit\_log
        \nodepart{second}
        \underline{id: UUID} \\
        table\_name: VARCHAR \\
        operation: VARCHAR \\
        changed\_at: TIMESTAMPTZ
    };

    \node[table, below=of checkins] (stats) {
        habit\_stats
        \nodepart{second}
        \underline{id: UUID} \\
        habit\_id: UUID (FK) \\
        total\_checkins: INTEGER \\
        updated\_at: TIMESTAMPTZ
    };

    \node[table, below=of schedules] (reminders) {
        reminders
        \nodepart{second}
        \underline{id: UUID} \\
        habit\_id: UUID (FK) \\
        reminder\_time: TIME \\
        is\_active: BOOLEAN
    };

    \node[table, below=of audit] (batch_jobs) {
        batch\_import\_jobs
        \nodepart{second}
        \underline{id: UUID} \\
        user\_id: UUID (FK) \\
        status: VARCHAR \\
        started\_at: TIMESTAMPTZ
    };

    \node[table, below=of batch_jobs] (batch_errors) {
        batch\_import\_errors
        \nodepart{second}
        \underline{id: UUID} \\
        job\_id: UUID (FK) \\
        error\_message: TEXT \\
        created\_at: TIMESTAMPTZ
    };

    \node[table, below=of habit_tags] (manual) {
        \_manual\_migrations
        \nodepart{second}
        \underline{id: UUID} \\
        name: VARCHAR \\
        status: VARCHAR \\
        applied\_at: TIMESTAMPTZ
    };

    \draw[edge] (users) -- node[midway, above] {1:1} (profiles);
    \draw[edge] (users) -- node[midway, above] {1:N} (habits);
    \draw[edge] (users) -- node[midway, left] {1:N} (checkins);
    \draw[edge] (users) -- node[midway, left] {1:N} (batch_jobs);
    \draw[edge] (users) -- node[midway, left] {1:N} (audit);
    \draw[edge] (habits) -- node[midway, right] {1:N} (checkins);
    \draw[edge] (habits) -- node[midway, right] {1:N} (schedules);
    \draw[edge] (habits) -- node[midway, right] {1:N} (reminders);
    \draw[edge] (habits) -- node[midway, right] {1:1} (stats);
    \draw[edge] (habits) -- node[midway, right] {1:N} (habit_tags);
    \draw[edge] (tags) -- node[midway, right] {1:N} (habit_tags);
    \draw[edge] (batch_jobs) -- node[midway, left] {1:N} (batch_errors);
\end{tikzpicture}}
\caption{Логическая схема базы данных}
\label{fig:db-schema}
\end{figure}

\newpage
\section*{ПРИЛОЖЕНИЕ Б Примеры SQL-скриптов}
\addcontentsline{toc}{section}{ПРИЛОЖЕНИЕ Б Примеры SQL-скриптов}
В приложении приведены ссылки на основные SQL-скрипты проекта:
\begin{itemize}
  \item \texttt{packages/db/sql/001\_triggers.sql} --- триггеры аудита и агрегатов;
  \item \texttt{packages/db/sql/002\_functions.sql} --- скалярные и табличные функции;
  \item \texttt{packages/db/sql/003\_views.sql} --- представления для аналитики;
  \item \texttt{packages/db/sql/004\_indexes.sql} --- индексы для оптимизации запросов.
\end{itemize}

\newpage
\section*{ПРИЛОЖЕНИЕ В Swagger документация API}
\addcontentsline{toc}{section}{ПРИЛОЖЕНИЕ В Swagger документация API}
\setcounter{figure}{0}
\renewcommand{\thefigure}{В.\arabic{figure}}

\begin{figure}[H]
\centering
\fbox{\parbox{0.9\linewidth}{\centering Скриншот Swagger UI для API проекта.}}
\caption{Интерфейс Swagger UI}
\label{fig:swagger}
\end{figure}

\noindent Основные группы эндпойнтов представлены в разделе 2.6. Документация доступна по адресу \texttt{http://localhost:3001/api/docs}.

\newpage
\section*{ПРИЛОЖЕНИЕ Г Результаты EXPLAIN ANALYZE}
\addcontentsline{toc}{section}{ПРИЛОЖЕНИЕ Г Результаты EXPLAIN ANALYZE}
Полные планы выполнения запросов приведены в файле \texttt{docs/perf.md}. Основные результаты оптимизации:
\begin{itemize}
  \item история отметок пользователя за период: 145.456 мс \(\rightarrow\) 2.789 мс (52x быстрее);
  \item журнал аудита за 24 часа: 89.456 мс \(\rightarrow\) 1.289 мс (69x быстрее);
  \item среднее ускорение ключевых запросов: около 60x.
\end{itemize}

\end{document}
