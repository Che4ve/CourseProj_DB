\documentclass[a4paper,14pt]{extarticle}

\usepackage[utf8]{inputenc}
\usepackage[T2A]{fontenc}
\usepackage{tempora}
\usepackage[russian]{babel}
\usepackage{cmap}
\usepackage{geometry}
\usepackage{setspace}
\usepackage{indentfirst}
\usepackage{enumitem}
\usepackage{graphicx}
\usepackage{float}
\usepackage{array}
\usepackage{tabularx}
\usepackage{booktabs}
\usepackage{longtable}
\usepackage{xcolor}
\usepackage{listings}
\usepackage{caption}
\usepackage[hidelinks]{hyperref}
\usepackage{tikz}
\usetikzlibrary{positioning,arrows.meta,shapes.multipart,calc,shadows}
\usepackage{titlesec}
\usepackage{fancyhdr}

\geometry{
  left=30mm,
  right=15mm,
  top=20mm,
  bottom=20mm
}

\onehalfspacing
\setlength{\parindent}{1.25cm}
\setlength{\parskip}{0pt}
\sloppy

\setlist[itemize]{label=--, leftmargin=1.25cm, itemsep=0pt, topsep=0pt}
\setlist[enumerate]{label=\arabic*), leftmargin=1.25cm, itemsep=0pt, topsep=0pt}

\titleformat{\section}{\normalfont\Large\bfseries\centering}{\thesection}{1em}{\MakeUppercase}
\titleformat{\subsection}{\normalfont\large\bfseries}{\thesubsection}{1em}{}
\titleformat{\subsubsection}{\normalfont\normalsize\bfseries}{\thesubsubsection}{1em}{}

\DeclareCaptionFont{singlespacing}{\setstretch{1}}
\captionsetup{labelsep=emdash, font=singlespacing}

\renewcommand{\contentsname}{СОДЕРЖАНИЕ}
\renewcommand{\lstlistingname}{Листинг}

\lstset{
  language=SQL,
  basicstyle=\ttfamily\small,
  keywordstyle=\color{blue},
  commentstyle=\color{gray},
  stringstyle=\color{teal},
  numbers=left,
  numberstyle=\tiny,
  numbersep=6pt,
  stepnumber=1,
  frame=single,
  breaklines=true,
  showstringspaces=false,
  columns=fullflexible,
  tabsize=2
}

\pagestyle{fancy}
\fancyhf{}
\fancyfoot[C]{\thepage}
\renewcommand{\headrulewidth}{0pt}

\tikzset{
  block/.style={draw, rounded corners, align=center, minimum height=1.1cm, minimum width=3.4cm},
  arrow/.style={-{Latex[length=3mm]}, thick}
}

\begin{document}
\pagenumbering{arabic}

\begin{titlepage}
\thispagestyle{empty}
\begin{singlespace}

% Верхняя часть - информация об университете
\begin{center}
МИНИСТЕРСТВО НАУКИ И ВЫСШЕГО ОБРАЗОВАНИЯ РОССИЙСКОЙ ФЕДЕРАЦИИ \\
ФЕДЕРАЛЬНОЕ ГОСУДАРСТВЕННОЕ БЮДЖЕТНОЕ ОБРАЗОВАТЕЛЬНОЕ УЧРЕЖДЕНИЕ ВЫСШЕГО ОБРАЗОВАНИЯ \\[0.5cm]
\textbf{«МОСКОВСКИЙ АВИАЦИОННЫЙ ИНСТИТУТ \\
(национальный исследовательский университет)» (МАИ)} \\[0.5cm]
Институт № 8 «Компьютерные науки и прикладная математика» \\
Кафедра 806 «Вычислительная математика и программирование»
\end{center}

\vspace{40mm}

% Средняя часть - название работы и тема
\begin{center}
\textbf{КУРСОВОЙ ПРОЕКТ} \\[0.5cm]
по дисциплине «Базы данных» \\[0.5cm]
\textbf{Тема: «Информационная система для трекинга привычек»}
\end{center}

\vfill

% Правая часть - табличка с данными
\begin{flushright}
\begin{tabular}{ll}
Студент: & Чесноков В.Д. \\
Группа: & М8О-309Б-23 \\
Преподаватель: & Грубенко М.Д. \\
Оценка: & \underline{\hspace{3cm}} \\
Дата: & \underline{\hspace{3cm}} \\
Подпись: & \underline{\hspace{3cm}} \\
\end{tabular}
\end{flushright}

\vspace{20mm}

% Нижняя часть - город и год
\begin{center}
Москва 2025
\end{center}

\end{singlespace}
\end{titlepage}

\newpage
\tableofcontents

\newpage
\section*{РЕФЕРАТ}
\addcontentsline{toc}{section}{РЕФЕРАТ}
Отчет содержит 13 с., 2 рис., 14 табл., 5 источников.

Ключевые слова: БАЗА ДАННЫХ, POSTGRESQL, NESTJS, PRISMA, ТРЕКЕР ПРИВЫЧЕК, АУДИТ, ТРИГГЕРЫ, BATCH IMPORT.

Объектом исследования является информационная система для трекинга привычек пользователей.

Цель работы --- проектирование и реализация полнофункциональной реляционной базы данных, интегрированной с серверным API, обеспечивающей хранение данных о привычках, их выполнение, аналитику и аудит изменений.

В результате работы спроектирована структура БД, реализована бизнес-логика на стороне СУБД (триггеры, функции), разработаны механизмы аудита и аналитической отчетности. Обеспечена высокая производительность через индексацию.

\newpage
\section*{ВВЕДЕНИЕ}
\addcontentsline{toc}{section}{ВВЕДЕНИЕ}
Современные сервисы трекинга привычек требуют надежного хранения данных, гибкой аналитики и строгого контроля изменений. Целью данного курсового проекта является разработка информационной системы с реляционной БД и серверным API, поддерживающими учет привычек и аналитическую обработку данных.

Для достижения цели решены следующие задачи:
\begin{enumerate}
  \item Проведен анализ предметной области.
  \item Спроектирована структура БД.
  \item Разработаны SQL-объекты: триггеры, ограничения, функции и представления.
  \item Реализован REST API и механизм массового импорта данных.
  \item Выполнена оптимизация запросов и анализ производительности.
\end{enumerate}

\newpage
\section{АНАЛИТИЧЕСКАЯ ЧАСТЬ}
\subsection{Обзор предметной области}
Предметная область системы --- управление личной эффективностью через трекинг привычек. Основными сущностями являются пользователи, их привычки, расписания выполнения, отметки (чекины) и теги. Система должна поддерживать учет прогресса, напоминания и аудит всех изменений.

\subsection{Постановка задачи}
В соответствии с техническим заданием необходимо разработать систему, удовлетворяющую требованиям:
\begin{itemize}
    \item Хранение данных в 10+ взаимосвязанных таблицах.
    \item Реализация бизнес-логики на уровне СУБД (триггеры, функции).
    \item Автоматический аудит всех изменений в ключевых таблицах.
    \item Обеспечение высокой производительности благодаря индексации.
    \item Реализация API с поддержкой массового импорта.
\end{itemize}

\newpage
\section{ПРОЕКТНАЯ ЧАСТЬ}
\subsection{Архитектура системы}
Система реализована как монорепозиторий. Backend построен на NestJS и взаимодействует с PostgreSQL 16 через Prisma ORM. Frontend реализован на Next.js.

\begin{figure}[H]
\centering
\begin{tikzpicture}[node distance=2.2cm]
\node[block] (frontend) {Frontend\\Next.js};
\node[block, right=of frontend] (backend) {Backend API\\NestJS + Prisma};
\node[block, right=of backend] (db) {PostgreSQL 16\\База данных};
\draw[arrow] (frontend) -- (backend);
\draw[arrow] (backend) -- (db);
\end{tikzpicture}
\caption{Архитектура информационной системы}
\label{fig:architecture}
\end{figure}

\subsection{Проектирование базы данных}
Схема БД включает в себя следующие таблицы:

\begin{table}[H]
\caption{Описание атрибутов таблицы \texttt{users}}
\centering
\small
\begin{tabular}{|p{4cm}|p{3cm}|p{8cm}|}
\hline
\textbf{Имя поля} & \textbf{Тип данных} & \textbf{Описание} \\ \hline
\texttt{id} & UUID & Первичный ключ. \\ \hline
\texttt{email} & TEXT & Уникальный email пользователя. \\ \hline
\texttt{password\_hash} & TEXT & Хеш пароля. \\ \hline
\texttt{full\_name} & TEXT & Полное имя. \\ \hline
\texttt{role} & VARCHAR(50) & Роль пользователя (user, admin). \\ \hline
\texttt{is\_active} & BOOLEAN & Статус активности аккаунта. \\ \hline
\end{tabular}
\end{table}

\begin{table}[H]
\caption{Описание атрибутов таблицы \texttt{user\_profiles}}
\centering
\small
\begin{tabular}{|p{4cm}|p{3cm}|p{8cm}|}
\hline
\textbf{Имя поля} & \textbf{Тип данных} & \textbf{Описание} \\ \hline
\texttt{id} & UUID & Первичный ключ. \\ \hline
\texttt{user\_id} & UUID & Внешний ключ к \texttt{users}. \\ \hline
\texttt{bio} & TEXT & О себе. \\ \hline
\texttt{timezone} & VARCHAR(100) & Часовой пояс пользователя. \\ \hline
\texttt{date\_of\_birth} & DATE & Дата рождения. \\ \hline
\texttt{theme\_preference} & SMALLINT & Предпочтения темы оформления. \\ \hline
\end{tabular}
\end{table}

\begin{table}[H]
\caption{Описание атрибутов таблицы \texttt{habits}}
\centering
\small
\begin{tabular}{|p{4cm}|p{3cm}|p{8cm}|}
\hline
\textbf{Имя поля} & \textbf{Тип данных} & \textbf{Описание} \\ \hline
\texttt{id} & UUID & Первичный ключ. \\ \hline
\texttt{user\_id} & UUID & Владелец привычки. \\ \hline
\texttt{name} & TEXT & Название привычки. \\ \hline
\texttt{type} & VARCHAR(10) & Тип: 'good' или 'bad'. \\ \hline
\texttt{priority} & SMALLINT & Приоритет выполнения (0-10). \\ \hline
\texttt{is\_archived} & BOOLEAN & Флаг архивной привычки. \\ \hline
\end{tabular}
\end{table}

\begin{table}[H]
\caption{Описание атрибутов таблицы \texttt{habit\_schedules}}
\centering
\small
\begin{tabular}{|p{4cm}|p{3cm}|p{8cm}|}
\hline
\textbf{Имя поля} & \textbf{Тип данных} & \textbf{Описание} \\ \hline
\texttt{id} & UUID & Первичный ключ. \\ \hline
\texttt{habit\_id} & UUID & Внешний ключ к \texttt{habits}. \\ \hline
\texttt{frequency\_type} & VARCHAR(20) & Тип частоты (daily, weekly). \\ \hline
\texttt{weekdays\_mask} & SMALLINT & Битовая маска дней недели. \\ \hline
\texttt{start\_date} & DATE & Дата начала действия расписания. \\ \hline
\end{tabular}
\end{table}

\begin{table}[H]
\caption{Описание атрибутов таблицы \texttt{habit\_checkins}}
\centering
\small
\begin{tabular}{|p{4cm}|p{3cm}|p{8cm}|}
\hline
\textbf{Имя поля} & \textbf{Тип данных} & \textbf{Описание} \\ \hline
\texttt{id} & UUID & Первичный ключ. \\ \hline
\texttt{habit\_id} & UUID & Внешний ключ к \texttt{habits}. \\ \hline
\texttt{checkin\_date} & DATE & Дата выполнения привычки. \\ \hline
\texttt{checkin\_time} & TIME & Время выполнения. \\ \hline
\texttt{mood\_rating} & SMALLINT & Оценка настроения (1-5). \\ \hline
\texttt{notes} & TEXT & Заметки к выполнению. \\ \hline
\end{tabular}
\end{table}

\begin{table}[H]
\caption{Описание атрибутов таблицы \texttt{tags}}
\centering
\small
\begin{tabular}{|p{4cm}|p{3cm}|p{8cm}|}
\hline
\textbf{Имя поля} & \textbf{Тип данных} & \textbf{Описание} \\ \hline
\texttt{id} & UUID & Первичный ключ. \\ \hline
\texttt{name} & VARCHAR(100) & Название тега. \\ \hline
\texttt{slug} & TEXT & URL-friendly имя тега. \\ \hline
\texttt{color} & VARCHAR(7) & HEX-код цвета. \\ \hline
\texttt{usage\_count} & INTEGER & Счетчик использований тега. \\ \hline
\end{tabular}
\end{table}

\begin{table}[H]
\caption{Описание атрибутов таблицы \texttt{habit\_tags}}
\centering
\small
\begin{tabular}{|p{4cm}|p{3cm}|p{8cm}|}
\hline
\textbf{Имя поля} & \textbf{Тип данных} & \textbf{Описание} \\ \hline
\texttt{id} & UUID & Первичный ключ. \\ \hline
\texttt{habit\_id} & UUID & Ссылка на привычку. \\ \hline
\texttt{tag\_id} & UUID & Ссылка на тег. \\ \hline
\texttt{is\_primary} & BOOLEAN & Флаг основного тега. \\ \hline
\end{tabular}
\end{table}

\begin{table}[H]
\caption{Описание атрибутов таблицы \texttt{reminders}}
\centering
\small
\begin{tabular}{|p{4cm}|p{3cm}|p{8cm}|}
\hline
\textbf{Имя поля} & \textbf{Тип данных} & \textbf{Описание} \\ \hline
\texttt{id} & UUID & Первичный ключ. \\ \hline
\texttt{habit\_id} & UUID & Ссылка на привычку. \\ \hline
\texttt{reminder\_time} & TIME & Время напоминания. \\ \hline
\texttt{days\_of\_week} & SMALLINT & Маска дней недели для напоминания. \\ \hline
\texttt{is\_active} & BOOLEAN & Статус активности напоминания. \\ \hline
\end{tabular}
\end{table}

\begin{table}[H]
\caption{Описание атрибутов таблицы \texttt{habit\_stats}}
\centering
\small
\begin{tabular}{|p{4cm}|p{3cm}|p{8cm}|}
\hline
\textbf{Имя поля} & \textbf{Тип данных} & \textbf{Описание} \\ \hline
\texttt{habit\_id} & UUID & Ссылка на привычку. \\ \hline
\texttt{total\_checkins} & INTEGER & Всего выполнений. \\ \hline
\texttt{current\_streak} & INTEGER & Текущая серия (дней подряд). \\ \hline
\texttt{longest\_streak} & INTEGER & Рекордная серия. \\ \hline
\texttt{completion\_rate} & NUMERIC & Процент выполнения. \\ \hline
\end{tabular}
\end{table}

\begin{table}[H]
\caption{Описание атрибутов таблицы \texttt{audit\_log}}
\centering
\small
\begin{tabular}{|p{4cm}|p{3cm}|p{8cm}|}
\hline
\textbf{Имя поля} & \textbf{Тип данных} & \textbf{Описание} \\ \hline
\texttt{id} & UUID & Первичный ключ. \\ \hline
\texttt{table\_name} & TEXT & Имя затронутой таблицы. \\ \hline
\texttt{operation} & VARCHAR(10) & Тип операции (INSERT, UPDATE, DELETE). \\ \hline
\texttt{old\_data} & JSONB & Состояние записи до изменений. \\ \hline
\texttt{new\_data} & JSONB & Состояние записи после изменений. \\ \hline
\texttt{ip\_address} & TEXT & IP-адрес инициатора. \\ \hline
\end{tabular}
\end{table}

\begin{table}[H]
\caption{Описание атрибутов таблицы \texttt{batch\_import\_jobs}}
\centering
\small
\begin{tabular}{|p{4cm}|p{3cm}|p{8cm}|}
\hline
\textbf{Имя поля} & \textbf{Тип данных} & \textbf{Описание} \\ \hline
\texttt{id} & UUID & Первичный ключ задачи. \\ \hline
\texttt{status} & VARCHAR & Статус (pending, processing, completed). \\ \hline
\texttt{total\_records} & INTEGER & Всего записей в батче. \\ \hline
\texttt{success\_count} & INTEGER & Успешно импортировано. \\ \hline
\texttt{error\_count} & INTEGER & Количество ошибок. \\ \hline
\end{tabular}
\end{table}

\begin{table}[H]
\caption{Описание атрибутов таблицы \texttt{batch\_import\_errors}}
\centering
\small
\begin{tabular}{|p{4cm}|p{3cm}|p{8cm}|}
\hline
\textbf{Имя поля} & \textbf{Тип данных} & \textbf{Описание} \\ \hline
\texttt{id} & UUID & Первичный ключ ошибки. \\ \hline
\texttt{job\_id} & UUID & Ссылка на задачу импорта. \\ \hline
\texttt{row\_number} & INTEGER & Номер строки с ошибкой. \\ \hline
\texttt{error\_message} & TEXT & Текст ошибки. \\ \hline
\end{tabular}
\end{table}


\subsection{Ограничения целостности и бизнес-логика}
Для обеспечения корректности данных применены:
\begin{itemize}
    \item \texttt{FOREIGN KEY} с правилами \texttt{ON DELETE CASCADE}.
    \item \texttt{CHECK} ограничения на тип привычки, рейтинг настроения (1-5) и приоритет.
    \item \texttt{UNIQUE} индекс на пару \texttt{(habit\_id, checkin\_date)}.
\end{itemize}

Бизнес-логика реализована через:
\begin{itemize}
    \item \textbf{Триггеры:} автоматическое ведение \texttt{audit\_log} и обновление \texttt{habit\_stats} при каждом чекине.
    \item \textbf{Функции:} скалярная \texttt{calc\_completion\_rate} и табличная \texttt{report\_user\_habits} для формирования отчетов.
    \item \textbf{Представления:} \texttt{v\_daily\_completion} для визуализации активности.
\end{itemize}

\subsection{Оптимизация производительности}
Для ускорения работы созданы B-tree индексы на некоторых полях. Результаты анализа через \texttt{EXPLAIN ANALYZE} представлены в таблице \ref{tab:perf}.

\begin{table}[H]
\caption{Результаты оптимизации запросов}
\label{tab:perf}
\centering
\begin{tabular}{l c c c}
\toprule
Запрос & До, мс & После, мс & Ускорение \\
\midrule
Фильтрация привычек по пользователю & 1.463 & 0.049 & $\approx$ 30x \\
\bottomrule
\end{tabular}
\end{table}

\newpage
\section{ТЕХНОЛОГИЧЕСКАЯ ЧАСТЬ}
\subsection{Контейнеризация и развертывание}
Развертывание системы выполняется через Docker Compose. Стек включает PostgreSQL 16, NestJS Backend и Next.js Frontend.

\subsection{Массовая загрузка данных}
Реализован механизм Batch Import, позволяющий загружать тысячи записей через JSON/CSV. Система обеспечивает транзакционность: ошибки валидации логируются в \texttt{batch\_import\_errors}, не прерывая загрузку корректных данных.

\subsection{Тестирование}
Для проверки производительности БД заполнена тестовыми данными с помощью Faker.
\begin{table}[H]
\caption{Объем тестовых данных}
\centering
\begin{tabular}{l r}
\toprule
Сущность & Количество записей \\
\midrule
Пользователи & 100 \\
Привычки & 800 \\
Отметки выполнения & 10 000 \\
Теги и связи & 1 500 \\
\bottomrule
\end{tabular}
\end{table}

\newpage
\section*{ЗАКЛЮЧЕНИЕ}
\addcontentsline{toc}{section}{ЗАКЛЮЧЕНИЕ}
В ходе проекта разработана информационная система для трекинга привычек. Спроектированная БД обеспечивает надежное хранение и аудит данных. Использование триггеров и функций позволило вынести часть бизнес-логики на сторону СУБД, а индексация обеспечила высокую скорость работы под нагрузкой.

\newpage
\addcontentsline{toc}{section}{СПИСОК ИСПОЛЬЗОВАННЫХ ИСТОЧНИКОВ}
\begin{thebibliography}{5}
  \bibitem{postgres} PostgreSQL 16 Documentation [Электронный ресурс]. --- URL: \url{https://www.postgresql.org/docs/16/}.
  \bibitem{nestjs} NestJS Documentation [Электронный ресурс]. --- URL: \url{https://docs.nestjs.com/}.
  \bibitem{prisma} Prisma Documentation [Электронный ресурс]. --- URL: \url{https://www.prisma.io/docs}.
  \bibitem{nextjs} Next.js Documentation [Электронный ресурс]. --- URL: \url{https://nextjs.org/docs}.
  \bibitem{docker} Docker Documentation [Электронный ресурс]. --- URL: \url{https://docs.docker.com/}.
\end{thebibliography}

\newpage
\section*{ПРИЛОЖЕНИЕ А Логическая схема базы данных}
\addcontentsline{toc}{section}{ПРИЛОЖЕНИЕ А Логическая схема базы данных}
\setcounter{figure}{0}
\renewcommand{\thefigure}{А.\arabic{figure}}

\begin{figure}[H]
\centering
\resizebox{\textwidth}{!}{%
\begin{tikzpicture}[
    node distance=1.5cm,
    table/.style={
        rectangle split, rectangle split parts=2,
        draw, fill=white, rounded corners,
        minimum width=2.5cm, font=\tiny\ttfamily,
        align=left, drop shadow
    },
    edge/.style={->, >=Latex, thick, gray!60},
    rel/.style={midway, fill=white, inner sep=1pt, font=\tiny\sffamily, text=black}
]

    % --- ROW 1: CORE USERS & SYSTEM ---
    
    \node[table] (users) {
        \nodepart{one} \textbf{users}
        \nodepart{second}
        \underline{id: UUID} \\
        email: TEXT \\
        password\_hash: TEXT \\
        full\_name: TEXT \\
        role: VARCHAR(50) \\
        is\_active: BOOLEAN \\
        created\_at: TIMESTAMPTZ \\
        login\_count: INT
    };

    \node[table, left=2.0cm of users] (profiles) {
        \nodepart{one} \textbf{user\_profiles}
        \nodepart{second}
        \underline{id: UUID} \\
        user\_id: UUID (FK) \\
        bio: TEXT \\
        avatar\_url: TEXT \\
        timezone: VARCHAR(100) \\
        date\_of\_birth: DATE \\
        theme\_preference: SMALLINT \\
        notification\_enabled: BOOL
    };

    \node[table, right=1.5cm of users] (batch_jobs) {
        \nodepart{one} \textbf{batch\_import\_jobs}
        \nodepart{second}
        \underline{id: UUID} \\
        user\_id: UUID (FK) \\
        entity\_type: VARCHAR \\
        status: VARCHAR \\
        total\_records: INT \\
        success\_count: INT \\
        error\_count: INT \\
        progress\_percent: DECIMAL \\
        started\_at: TIMESTAMPTZ \\
        completed\_at: TIMESTAMPTZ
    };

    % --- ROW 2: AUDIT & BATCH DETAILS ---
    
    \node[table, below=1.0cm of users] (audit) {
        \nodepart{one} \textbf{audit\_log}
        \nodepart{second}
        \underline{id: UUID} \\
        table\_name: TEXT \\
        operation: VARCHAR \\
        record\_id: UUID \\
        user\_id: UUID \\
        old\_data: JSONB \\
        new\_data: JSONB \\
        ip\_address: VARCHAR \\
        changed\_at: TIMESTAMPTZ
    };

    \node[table, below=1.0cm of batch_jobs] (batch_errors) {
        \nodepart{one} \textbf{batch\_import\_errors}
        \nodepart{second}
        \underline{id: UUID} \\
        job\_id: UUID (FK) \\
        row\_number: INT \\
        record\_data: JSONB \\
        error\_message: TEXT \\
        error\_code: VARCHAR \\
        created\_at: TIMESTAMPTZ
    };

    % --- ROW 3: HABITS CORE (Balanced spacing) ---

    \node[table, below=1.0cm of profiles] (checkins) {
        \nodepart{one} \textbf{habit\_checkins}
        \nodepart{second}
        \underline{id: UUID} \\
        habit\_id: UUID (FK) \\
        user\_id: UUID (FK) \\
        checkin\_date: DATE \\
        checkin\_time: TIME \\
        mood\_rating: SMALLINT \\
        duration\_minutes: INT \\
        notes: TEXT \\
        created\_at: TIMESTAMPTZ
    };
    
    \node[table, below=1.0cm of checkins] (habits) {
        \nodepart{one} \textbf{habits}
        \nodepart{second}
        \underline{id: UUID} \\
        user\_id: UUID (FK) \\
        name: TEXT \\
        description: TEXT \\
        type: VARCHAR(10) \\
        priority: SMALLINT \\
        is\_archived: BOOLEAN \\
        display\_order: INT \\
        created\_at: TIMESTAMPTZ
    };

    \node[table, below=1.0cm of audit] (habit_tags) {
        \nodepart{one} \textbf{habit\_tags}
        \nodepart{second}
        \underline{id: UUID} \\
        habit\_id: UUID (FK) \\
        tag\_id: UUID (FK) \\
        priority: SMALLINT \\
        is\_primary: BOOLEAN \\
        assigned\_by: VARCHAR(50) \\
        assigned\_at: TIMESTAMPTZ
    };
    
    % --- ROW 4: HABIT DETAILS & EXTENSIONS ---



    \node[table, below=1.5cm of habit_tags] (schedules) {
        \nodepart{one} \textbf{habit\_schedules}
        \nodepart{second}
        \underline{id: UUID} \\
        habit\_id: UUID (FK) \\
        frequency\_type: VARCHAR \\
        frequency\_value: INT \\
        weekdays\_mask: SMALLINT \\
        start\_date: DATE \\
        end\_date: DATE \\
        is\_active: BOOLEAN
    };
    
    \node[table, right=1.5cm of habit_tags] (tags) {
        \nodepart{one} \textbf{tags}
        \nodepart{second}
        \underline{id: UUID} \\
        name: VARCHAR(100) \\
        slug: TEXT \\
        color: VARCHAR(7) \\
        usage\_count: INT \\
        is\_system: BOOLEAN \\
        created\_at: TIMESTAMPTZ
    };

    \node[table, below=1.0cm of habits] (reminders) {
        \nodepart{one} \textbf{reminders}
        \nodepart{second}
        \underline{id: UUID} \\
        habit\_id: UUID (FK) \\
        reminder\_time: TIME \\
        days\_of\_week: SMALLINT \\
        notification\_text: TEXT \\
        delivery\_method: VARCHAR \\
        is\_active: BOOLEAN \\
        created\_at: TIMESTAMPTZ
    };

    \node[table, below=1.5cm of tags] (stats) {
        \nodepart{one} \textbf{habit\_stats}
        \nodepart{second}
        \underline{id: UUID} \\
        habit\_id: UUID (FK) \\
        total\_checkins: INT \\
        current\_streak: INT \\
        longest\_streak: INT \\
        completion\_rate: DECIMAL \\
        average\_mood: DECIMAL \\
        last\_checkin\_at: DATE \\
        updated\_at: TIMESTAMPTZ
    };

    % --- RELATIONS ---
    \draw[edge] (users) -- node[rel] {1:1} (profiles);
    % \draw[edge] (users) -- node[rel] {1:N} (habits);
    \draw[edge] (users) -- node[rel] {1:N} (batch_jobs);
    \draw[edge] (users) -- node[rel] {1:N} (audit);
    
    % Curved lines for far objects
    \draw[edge] (users.south west) to[out=240,in=60] node[rel] {1:N} (checkins.north east);
    % \draw[edge] (users.south) to[out=260,in=90] node[rel, pos=0.3] {1:N} (audit.north east);
    \draw[edge] (habits.east) to[out=260,in=140] node[rel, pos=0.3] {1:N} (stats.north west);    
    \draw[edge] (users.south west) to[out=260,in=60] node[rel, pos=0.3] {1:N} (habits.north east);

    \draw[edge] (habits) -- node[rel] {1:N} (checkins);
    \draw[edge] (habits) -- node[rel] {1:N} (schedules);
    \draw[edge] (habits) -- node[rel] {1:N} (habit_tags);
    \draw[edge] (habits) -- node[rel] {1:N} (reminders);
    
    % \draw[edge] (habits) to[out=250,in=60] node[rel] {1:1} (stats.north east);
    
    % % Multi-segment paths for clarity
    % \draw[edge] (habits.south west) .. controls +(-1,-1) and +(0,1) .. (reminders.north west);

    \draw[edge] (tags) -- node[rel] {1:N} (habit_tags);
    \draw[edge] (batch_jobs) -- node[rel] {1:N} (batch_errors);

\end{tikzpicture}}
\caption{Логическая схема базы данных}
\label{fig:db-schema}
\end{figure}

\end{document}
